\documentclass[]{article}
\usepackage{lmodern}
\usepackage{amssymb,amsmath}
\usepackage{ifxetex,ifluatex}
\usepackage{fixltx2e} % provides \textsubscript
\ifnum 0\ifxetex 1\fi\ifluatex 1\fi=0 % if pdftex
  \usepackage[T1]{fontenc}
  \usepackage[utf8]{inputenc}
\else % if luatex or xelatex
  \ifxetex
    \usepackage{mathspec}
    \usepackage{xltxtra,xunicode}
  \else
    \usepackage{fontspec}
  \fi
  \defaultfontfeatures{Mapping=tex-text,Scale=MatchLowercase}
  \newcommand{\euro}{€}
\fi
% use upquote if available, for straight quotes in verbatim environments
\IfFileExists{upquote.sty}{\usepackage{upquote}}{}
% use microtype if available
\IfFileExists{microtype.sty}{\usepackage{microtype}}{}
\ifxetex
  \usepackage[setpagesize=false, % page size defined by xetex
              unicode=false, % unicode breaks when used with xetex
              xetex]{hyperref}
\else
  \usepackage[unicode=true]{hyperref}
\fi
\hypersetup{breaklinks=true,
            bookmarks=true,
            pdfauthor={},
            pdftitle={},
            colorlinks=true,
            citecolor=blue,
            urlcolor=blue,
            linkcolor=magenta,
            pdfborder={0 0 0}}
\urlstyle{same}  % don't use monospace font for urls
\setlength{\parindent}{0pt}
\setlength{\parskip}{6pt plus 2pt minus 1pt}
\setlength{\emergencystretch}{3em}  % prevent overfull lines
\setcounter{secnumdepth}{0}


\begin{document}

\section{Fisiologia Vegetale}\label{fisiologia-vegetale}

La Fisiologia vegetale si interessa principalmente di ``meccanismi''.

Di che cos'ha bisogno una pianta per crescere?

\begin{itemize}
\itemsep1pt\parskip0pt\parsep0pt
\item
  Acqua;
\item
  luce;
\item
  temperature adatte (anche se in parte possono regolare la loro
  temperatura, in generale sono organismi che riflettono la temperatura
  dell'ambiente in cui vivono).
\end{itemize}

L'acqua è uno dei fattori più limitanti sul nostro pianeta. La maggior
parte delle piante non tollera bene la presenza di sali nell'acqua. Gli
ioni maggiormente utilizzati dalle piante sono gli K$^+$ e H$^+$ (questo
porta allo sviluppo di differenze di pH fondamentali per i processi),
mentre non tollerano bene il sodio.

La luce è un substrato complicato da trattare perché tutto ciò che
avviene grazie ad essa avviene in tempi brevissimi.

La luce ha due ruoli importanti per le piante:

\begin{itemize}
\itemsep1pt\parskip0pt\parsep0pt
\item
  permette la \textbf{fotosintesi} e dunque alle risorse energetiche (la
  fotosintesi è l'unico meccanismo attraverso il quale l'energia del
  sole viene resa disponibile per i processi metabolici);
\item
  regola il modo di crescere della pianta (ha un ruolo informativo
  sull'ambiente).
\end{itemize}

La luce induce delle risposte nella pianta grazie alla presenza di
recettori specifici (es. apertura/chiusura degli stomi grazie alle
tropine, curvatura del fusto, ecc.).

In fisiologia vegetale l'organismo modello è rappresentato da
\textbf{Arabidopsis} poichè questo organismo è eucariote, ha un genoma
piccolo e completamente sequenziato (5 cromosomi), e un ciclo vistale
breve (4 settimane).

In fisiologia vegetale viene utilizzato l'Agrobacterium tumefacens per
compiere della \emph{mutagenesi inserzionale}, ovvero per inserie nella
cellula vegetale delle nuove caratteristiche.

Tramite queste tecniche è poi possibile risalire ai geni che sono stati
inattivati dal DNA transfer inserito dall'Agrobacterium.

Le piante, da un punto di vista alimentare, sono organismi abbastanza
semplici: hanno bisogno di un po' di acqua, luce, sali minerali (piccole
molecole inorganiche) e CO$_2$.

Da un punto di vista cellulare, tissutale e sistemico le piante sono
molto più semplici degli animali.

Le radici hanno la funzione di ancorare la pianta al terreno, ed hanno
anche la funzione di trovare due nutrienti fondamentali per la pianta:
l'acqua e i sali minerali.

La luce e la CO$_2$ invece, sono assorbite delle foglie (e a volte dal
fusto). Le foglie sono anche il luogo in cui avviene la fotosintesi.

Le piante carnivore si sono adattate alla vita in terreni poveri di N
che recuperano dagli insetti che mangiano.

L'ossigeno può essere rappresentato come \textbf{ossigeno di singoletto
($^1$O$_2$)} o \textbf{ossigeno di tripletto ($^3$O$_2$)} e sono
responsabili per la formazione delle \textbf{ROS (reactive oxygen
species)}. La reattività di queste due molecole è molto diversa: qualla
di singoletto è molto alta, mentre quella di tripletto è bassa (è
l'ossigeno che respiriamo).

Le piante sono produttrici di $^1$O$_2$.

La clorofilla (\emph{chl}) è verde perché assorbe la luce blu e la rossa
mentre riflette quella verde.

Cosa vuol dire che la molecola \emph{assorbe} la luce?

Nel cloroplasto, l'energia luminosa è raccolta da due unità
funzionalmente diverse definite \emph{fotosistemi}. L'energia luminosa
assorbita è utilizzata per innescare il trasferimento di elettroni
tramite una serie di composti che fungono da donatori e da accettori di
elettroni.

La luce possiede proprietà sia di \emph{particella} che di \emph{onda}.
Un'onda è caratterizzata da una \textbf{lunghezza d'onda}, definita
dalla lettera greca \emph{($\lambda$)} che è la distanza fra due picchi
successivi nell'onda. La \textbf{frequenza}, rappresentata dalla lettera
greca \emph{ni ($\nu$)} è il numero di picchi d'onda che intercorrono in
un determinato tempo.

Esiste un'equazione che mette in relazione la lunghezza d'onda, la
frequenza e la velocità di qualsiasi onda:

\textbf{c=$\lambda$$\nu$}

dove \emph{c} è la velocità dell'onda (nel nostro caso la velocità della
luce, ovvero 3x10$^8$ m s$^-$$^1$)

La luce è anche una particella chiamata \textbf{fotone}. Ogni fotone
contiene una determinata quantità d'energia chiamata \textbf{quanto}.
L'energia (E) di un fotone dipende dalla frequenza della luce secondo la
relazione nota come *legge di Planck**:

\textbf{E=h$\nu$}

dove \emph{h} è la costante di Planck (6,626 x 10$\textsuperscript{-34}$
J s).

La luce del sole è come una pioggia di fotoni con frequenze diverse. I
nostri occhi sono sensibili solo a una piccola gamma di frequenze (circa
da 400 a 700 nm) dette complessivamente \emph{``spettro del visibile''}.
Lunghezze d'onda più corte fanno parte dello spettro dell'ultravioletto,
mentre frequenze più alte fanno parte dello spettro dell'infrarosso.

Lo \textbf{spettro di assorbimento} fornisce informazioni circa la
quantità di \textbf{energia luminosa} catturata o assorbita da una
molecola o sostanza in funzione della lunghezza d'onda.

Sfruttando lo spettro di assorbimento è possibile misurare la
concentrazione di una sostanza grazie alla \textbf{spettrofotometria di
massa}. In questa tecnica si misura e viene utilizzata la quantità di
luce che un campione assorbe ad una particolare lunghezza d'onda per
determinare la concentrazione del campione, confrontandolo con dati di
riferimento di uno standard appropiato. La misura più utile
dell'assorbimento della luce è l'\textbf{assorbanza (A)}. L'assorbanza è
definita come \textbf{A= logI$_0$ / I} dove I$_0$ è l'intensità della
luce che incide sul campione e I è l'intensità della luce trasmessa dal
campione. L'assorbanza di un campione può essere messa in relaizone alla
concentrazione della specie assorbente attraverso l'utilizzo della
\textbf{legge di Beer}:

\textbf{A=$\varepsilon$ cl}

Dove \emph{c} è la concentrazione, misurata solitamente in moli per
litro; \emph{l} è la lunghezza del cammino ottimo, solitamente di 1 cm;
$\varepsilon$ è una costante di proporzionalità come coefficiente di
estinzione molare, con unità espresse solitamente in litri per mole per
centimetro. Il valore di $\varepsilon$ è in funzione sia del particolare
composto che è misurato che della lunghezza d'onda. Nella clorofilla
$\epsilon$ è 65000 unità di assorbanza/moli; la luce che esce da una
soluzione 1 molare di clorofilla sarebbe 10$\textsuperscript{-65000}$.

Le componenti dello \textbf{spettrofotometro} sono:

\begin{itemize}
\itemsep1pt\parskip0pt\parsep0pt
\item
  una sorgente luminosa;
\item
  un'apparecchiatura che seleziona la lunghezza d'onda come un
  monocromatore o un filtro;
\item
  una camera che contiene il campione;
\item
  un rilevatore di luce;
\item
  un sistema di lettura.
\end{itemize}

La clorofilla appare verde ai nostri occhi poichè assorbe la luce nelle
regioni del rosso e del blu dello spettro, mentre riflette la luce
verde.

L'assorbimento della luce blu eccita la molecola di clorofilla
maggiormente rispetto all'assorbiemento della luce rossa. nello stato di
eccitazione superiore, la clorofilla è estremamente instabile e cede
rapidamente un po' della sua energia all'ambiente circostante sotto
forma di calore, passando così a uno stato eccitato minore dove può
essere stabile per alcuni secondi (10$^-$$^9$ s).

\subsection{La fotosintesi}\label{la-fotosintesi}

La fotosintesi è un processo REDOX (trasferimento di elettroni).

L'ossigeno è un forte ossidante (tende a catturare elettroni) mentre
l'acqua per contro è un pessimo riducente.

La termodinamica è una scienza che non si interessa di cosa avviene
dentro gli oggetti che si studiano, ma si interessa di cosa entra e cosa
esce dal sistema; ciò significa che se l'oggetto fosse fatto di
ossigeno, acqua, NADP e NADPH, le energie coinvolte sarebbero date solo
dall'energia di quei potenziali.

Nella coppia NADP$^+$/NADPH il potenziale redox è di -420 mV: il NADPH è
un forte riducente mentre il NADP$^+$ è un debole ossidante.

Se io voglio trasportare un elettrone dall'acqua al NADP$^+$ devo
fornire un'energia pari alla differenza dei potenziali redox. Questa
energia viene fornita dalla luce.

Il ruolo della luce nella fotosintesi è fornire l'energia necessaria per
trasferire un elettrone dall'acqua al NADP$^+$.

Ci sono tre tipi di fotosintesi:

\begin{enumerate}
\def\labelenumi{\arabic{enumi}.}
\itemsep1pt\parskip0pt\parsep0pt
\item
  \textbf{fotosintesi ossigenica} (è la più conosciuta) dove viene
  utilizzata l'acqua come donatore di elettroni (riducente). Nel corso
  di questa reazione viene prodotto ossigeno ed elettroni che vengono
  trasportati fino al NADP$^+$. La reazione è H$_2$O + NADP$^+$
  $\leftrightarrow$ O$_2$ + NADPH. Perchè avvenga è necessario fornire
  energia che deriva dalla luce;
\item
  \textbf{fotosintesi anossigenica} dove non viene prodotto ossigeno
  (viene effettuata da alcuni batteri come i Pseudomonas);
\item
  il terzo tipo viene effettuata grazie alla \textbf{batterio
  rodopsina}.
\end{enumerate}

Le prime due fotosintesi sono basate sulla clorofilla.

Tutti gli organismi verdi eucarioti e i cianobatteri (o alche azzurre)
sono ossigenici, mentre i procarioti e gli alobatteri fanno la
fotosintesi anossigenica o utilizzano la rodopsina.

Il primo ad interessarsi al procesos fotosintetico fu Aristotele.

Il primo esperimento eseguito con metodo scientifico per dimostrare la
fotosintesi, invece, fu effettuato dal medico Jan Baptista van Helmont:
esso fece crescere un salice in un terreno precedentemente pesato. Dopo
5 anno pesò il salice ed in terreno scoprendo che il salice era
cresciuto di 70 kg, mentre il terreno era diminuito di 57 g. Da questo
dedusse che non era solo il terreno a nutrire la pianta.

Un altro studioso, Priestley, fece un altro esperimento: mise sotto una
campana di vetro un topolino e sotto un'altra una pianta di menta.
Voleva dimostrare che la vita di un animale è un processo redox, e che
le piante producono O$_2$.

I batteri fotosintetici (fotosintesi anossigenica) si dividono tra:

\begin{itemize}
\itemsep1pt\parskip0pt\parsep0pt
\item
  \textbf{sulfurei}, hanno bisogno di molecole ridotte dello zolfo come
  H$_2$S (n.o. dello S -2) che trasformano in zolfo elementare (S$_2$)
  liberando 2 elettroni che finiscono nella CO$_2$ (lo zolfo si comporta
  da riducente);
\item
  e \textbf{non sulfurei}.
\end{itemize}

Venne fatto un esperimento per capire se l'ossigeno liberato dalla
piante provenisse dall'acqua o dalla CO$_2$. Si marcarono prima molecole
di H$_2$O e poi di CO$_2$ e si scoprì che proveniva dall'acqua.

\subsubsection{I cloroplasti}\label{i-cloroplasti}

La fotosintesi avviene nei \textbf{cloroplasti}. Questi sono considerati
``avanzi di cianobatteri'' e si pensa che la loro origine sia di tipo
endosimbiontico.

I cloroplasti sono \emph{organelli semiautonomi} poichè contengono un
DNA proprio che gli permette di sintetizzare alcune proteine.

Appartengono al gruppo di organelli costituiti da una doppia unità di
membrana, detta \emph{envelope}, chiamati \textbf{plastidi}.

All'interno del cloroplasto si trova una matrice, che è lontana
dall'essere uniforme.

I cloroplasti presentano un ulteriore sistema di membrane chiamato
\textbf{tilacoidi}. Questo sistema di membrane, che può essere
immaginato come una sorta di sacchetto ripiegato numerosissime volte per
essere contenuto nel cloroplasto, separa una zona interna al tilacoide
ed una esterna. I tilacoidi sono numerosi e sono tutti collegati tra
loro tramite una lamella. L'impilamento dei tilacoidi origina un
\textbf{granum}. I grana sono collegati gli uni agli altri attraverso
tilacoidi non impilati definiti \textbf{lamelle stromatiche}.

Lo scomparto fluido che avvolge i tilacoidi è denominato \textbf{stroma}
ed ha una composiizone simile alla matrice dei mitocondri.

Da un punto di vista strutturale i tilacoidi sono abbastanza complicati.

Le clorofilla è associata alle membrane tilacoidali, e tutto ciò che ha
a che fare con la \emph{fase luminosa} della fotosintesi avviene sulle
\emph{membrane tilacoidali}. Tutto ciò che invece ha a che fare con la
\emph{fase buia} della fotosintesi avviene nella \emph{matrice
stromatica}.

\subsubsection{I pigmenti della pianta}\label{i-pigmenti-della-pianta}

L'energia della luce solare è assorbita principalmente dai pigmenti
della pianta. Tutti i pigmenti che sono attivi nella fotosintesi si
trovano nel cloroplasto.

Esistono diversi tipi di clorofille: le clorofille \emph{a} e \emph{b}
sono abbondanti nelle piante, mentre le clorofille \emph{c} e \emph{d}
si trovano nei protisti e nei cianobatteri.

Tutte le clorofille hanno una complessa struttura ad anello formata da:

\begin{itemize}
\itemsep1pt\parskip0pt\parsep0pt
\item
  un \textbf{tetrapirrolo};
\item
  un atomo di \textbf{magnesio (Mg$^2$$^+$)} al centro;
\item
  una \textbf{coda idrocarburica} (idrofoba) a 15 atomi di carbonio (un
  fitolo) che ancora la clorofilla alla porzione idrofoba dell'ambiente
  in cui è presente;
\item
  un \textbf{sostituente} (residuo etilenico). Se questo sostituente è
  un residuo formidico allora la clorofilla sarà di tipo \emph{b}.
\end{itemize}

La struttura ad anello contiene alcuni elettroni legati debolmente ed è
la parte della molecola che è coinvolta nella transizione elettronica e
nelle reazioni di redox.

Altri pigmenti fotosintetici sono i carotenoidi. Questi si dividono in:

\begin{itemize}
\itemsep1pt\parskip0pt\parsep0pt
\item
  \textbf{caroteni}, formati da carbonio e idrogeno. Sono tipicamente
  \emph{arancioni};
\item
  \textbf{xantofille}, formate da carbonio, idrogeno e ossigeno. Sono
  tipicamente \emph{gialle}.
\end{itemize}

I carotenoidi sono costituenti integrali delle memrbane tilacoidali e
sono spesos in stretto rapporto con molte proteine che costituiscono
l'apparato fotosintetico. L'energia luminosa assorbita dai carotenoidi è
trasferita alle clorofille per la fotosintesi; per questo motivo i
carotenoidi sono definiti \textbf{pigmenti accessori}.

\textbf{IMMAGINE p232}

La maggior parte dei pigmenti funziona da \textbf{complesso antenna},
captando la luce e trasferendo l'energia al \textbf{complesso del centro
di reazione}, dove avvengono le reazioni chimiche di ossidazione e
riduzione che portano all'accumulo a lungo termine dell'energia. Nella
fotosintesi ossigenica esistono 3 tipi di centri di reazione.

Con il termine \textbf{fotosistema} si intende il centro di reazione e
l'insieme dei pigmenti che lo servono come antenna.

Nella fotosintesi ossigenica esistono due centri di reazione:

\begin{enumerate}
\def\labelenumi{\arabic{enumi}.}
\itemsep1pt\parskip0pt\parsep0pt
\item
  \textbf{fotosistema I (PSI)}. Assorbe preferenzialmente la luce nel
  rosso lontano a lunghezze d'onda superiori a 680 nm, produce un forte
  agente riducente capace di ridurre il NADP$^+$ e un debole agente
  ossidante;
\item
  \textbf{fotosistema 2 (PSII)}. Assorbe preferenzialmente la luce rossa
  a 680 nm ed è poco stimolata dalla luce nel rosso lontano, produce un
  forte agente ossidante in grado di ossidare l'acqua e un debole agent
  eriducente.
\end{enumerate}

Questa denominazione dei fotosistemi rispecchia solamente l'ordine in
cui sono stati scoperti, infatti le reazioni avvengono prima nel PSII e
poi nel PSI.

Nel 1932 venne messo a punto un esperimento che fornì la prova della
cooperazione di numerose molecole di clorofilla nella conversione
dell'energia durante la fotosintesi. Gli studiosi fornirono dei lampi di
luce molto brevi a una sospensione di alche verdi e misurarono la
quantità di ossigeno prodotto. I lampi erano a intervalli di circa 0,1
s, un tempo che si era precedentemente determinato essere abbastanza
lungo da completare i passaggi enzimatici del processo prima dell'arrivo
del lampo successivo. Variando poi l'energia dei lampi scoprirono che ad
alta intensità, quando veniva fornito un lampo ancora più intenso, la
produzione di ossigeno non aumentava: il sistema fotosintetico era cioè
saturo di luce. In seguito a questo esperimento si constatò che in
condizioni saturanti veniva prodotta solo una molecola di ossigeno per
ogni 25000 molecole di clorofilla presenti nel campione. L'energia
luminosa richiesta per portare a termine la reazione fotosintetica è di
circa 9 o 10 fotoni.

Possono essere dunque disegnati dei grafici che tengono conto di quanto
ossigeno viene prodotto in funzione della densità di luce alla quale la
pianta viene esposta. Ad intensità basse si ha una \emph{relazione
lineare} tra l'intensità della luce e la quantità di ossigeno prodotta,
dove tutta la luce assorbita viene utilizzata per compiere la
fotosintesi (la lucein questo caso rappresenta un \emph{fattore
limitante}).

Oltre una certa intensità di radiazione invece si raggiunge una sorta di
plateau dove tutti gli enzimi a disposizione funzionano e si ottiene
infatti la \emph{velocità massima}, ovvero si ha una velocità di
reazione in cui il 100\% degli enzimi sono impegnati nella catalisi.

\textbf{IMMAGINE 7.11 p235}

La luce che viene assorbita e non viene utilizzata per la fotosintesi
viene degradata in calore a bassa temperatura (che è una forma di
energia bassa). Molta dell'energia viene degradata per via termica
perché altrimenti questo eccesso di energia potrebbe rappresentare un
fattore di stress pericoloso. Quindi le piante devono essere capaci di
regolare la luce che assorbono.

Un esperimento chiave per la comprensione del processo fotosintetico
risale agli anni '50; in questo esperimento venne misurata la velocità
di fotosintesi in funzione del tipo di luce che colpisce la pianta. La
velocità della fotosintesi venne misurata utilizzando \emph{luci rosso
chiare} e \emph{luci rosso scure}. Le due luci separatamente catalizzano
la fotosintesi più o meno con la stessa efficienza, me se si accendono
contemporaneamente le due luci si ha un'efficienza altissima (più alta
della somma dell'efficienza delle due luci separate) data dall'effetto
sinergico. Da questo si dedusse che probabilmente la fotosintesi
presenta due distinte reazioni di cui probabilmente una principalmente
sensibile alla luce rosso chiara e l'altra più sensibile a quella rosso
scura.

Negli anni è stata poi accertata la presenza di due fotosistemi.

Il PSII è localizzato prevalentemente nelle lamelle dei grana, mentre il
PSI è localizzato nelle lamelle stromatiche e ai bordi delle lamelle dei
grana. Queste diverse localizzazioni dei PS permettono la separazione
chimica dei due fotosistemi.

Il PSI e il PSII hanno caratteristiche di assorbimento distinte. Le
clorofille dei centri di reazione sono trasitoriamente ossidate dopo
aver perso un elettrone e prima di essere nuovamente ridotte. Nello
stato ossidato le clorofille perdono la loro caratteristica e la forte
capacità di assorbimento della luce nella regione rossa dello spettro,
diventando \emph{``sbiancate''}.

In seguito ad alcuni studi si è scoperto che la clorofilla del centro di
reazione del PSI aveva nel suo stato ridotto un massimo di assorbimento
a 700 nm; per questo motivo viene definita \textbf{P700} (P sta per
\emph{pigmento}). Il PSII invece ha un massimo di assorbimento a 680 nm
e per questo la sua clorofilla è conosciuta come \textbf{P680}. Nello
stato ossidato le clorofille die centri di reazione contengono un
elettrone spaiato

Oltre ai due PS concorre nel processo fotosintetico anche il
\textbf{complesso del citocromo B$_6$-f} della catena di trasporto degli
elettroni, che unisce i due fotosistemi. Questo complesso è equamente
distribuito fra le lamelle stromatiche e quelle granali. I due eventi
fotochimici che avvengono durante la fotosintesi che sviluppa O$_2$
risultano distintamente separati. Questa separazione implica che uno o
più \emph{carriers} di elettroni che operano fra i PS diffonda dalla
regione dei grama della membrana alla regione stromatica dove gli
elettroni sono ceduti al PSI.

\subsubsection{Organizzazione dei sistemi
antenna}\label{organizzazione-dei-sistemi-antenna}

La sequenza dei pigmenti all'interno dell'antenna che convoglia
l'energia assorbita verso il centro di reazione, presenta massimi di
assorbimento che sono progressivamente spostati verso lunghezze d'onda
rossa. Questo spostamento indica che l'energia dello stato di
eccitazione è in un certo modo più bassa vicino al centro di reazione di
quanto non lo sia nelle parti più periferiche del sistema antenna. Come
risultato di questo processo l'eccitazione viene trasferita a diverse
molecole in sequenza e una parte viene persa nell'ambiente sotto forma
di calore. Questo conferisce al sistema un certo grado di direzionalità,
poichè per tornare indietro occorrerebbe fornire nuovamente l'energia
persa.

Tra le proteine antenna più abbondanti, associate principalmente al PSII
per la cattura della luce, troviamo le \textbf{proteine LHCII}. Altre
sono associate al PSI e per questo dette LHCI.

La proteina LHCII possiede 3 regioni ad $\alpha$e-elica e lega 14
molecole di clorofilla \emph{a} e \emph{b} e 4 carotenoidi.

La luce assorbita dai carotenoidi o dalle clorofille \emph{b} è
velocemente trasferita alle clorofille \emph{a} e quindi ad altri
pigmenti antenna strettamente associati al centro di reazione.

\subsubsection{L'ossidazione dell'acqua tramite il fotosistema
II}\label{lossidazione-dellacqua-tramite-il-fotosistema-ii}

Il PSII è capace di catalizzare una reazione di questo tipo:

H$_2$O + \textbf{PQ} $\leftrightarrow$ O$_2$ + \textbf{PQH$_2$}

Dove PQ è un \textbf{plastochinone} (si trova nei cloroplasi) e PQH$_2$
è un \textbf{plastochinolo} o plastoidrochinone.

Il PSII è contenuto in un supercomplesso proteico costituito da molte
subunità tra le quali:

\begin{itemize}
\itemsep1pt\parskip0pt\parsep0pt
\item
  diverse proteine (circa 22);
\item
  clorofilla (circa 250 molecole di clorofilla);
\item
  una ventina di carotenoidi;
\item
  ioni metallici (come Ferro, Manganese, Cloro e Calcio)\ldots{}
\end{itemize}

Il fulcro del centro di reazione è formato da due proteine di membrana
conosciute come \textbf{D1} e \textbf{D2} alle quali troviano legata la
clorofilla come donatore primario, altre clorofille, carotenoidi, la
feofitina e i \emph{plastochinoni}.

La \textbf{proteina D1} è una delle prime proteine dei tilacoidi ad
essere stata identificata nel \textbf{gene PSBa}; questa proteina è una
delle poche il cui gene è rimasto nei cloroplasti. Questa proteina è
formata da \textbf{5 eliche transmembrana}: la parte amminoterminale è
esposta nello stroma del cloroplasto, mentre la parte C terminale è
esposta nel lume del cloroplasto. La proteina D1 ha un turnover elevato,
ed è una proteina facile da studiare.

Sia la proteina D1 che la D2, sono formate da una sequenza polipeptidica
di 353 amminoacidi e sono abbastanza simili da far pensare ad un evento
di duplicazione genica.

Le due proteine costituiscono così un \emph{eterodimero} che rappresenta
il core del PSII. Sono entrambe formate da \textbf{5 eliche di cui la
quarta e la quinta sono parzialmente sovrapposte}. E' nei punti di
sovrapposizione che si trovano gli elementi importanti per la
fotosintesi; qui è presente una molecola di clorofilla, chiamata
\textbf{P680}, che quando assorbe la luce opera la separazione delle
cariche e rappresenta il centro di reazione del fotosistema.

Quando il centro di reazione assorbe la luce succede che un elettrone
del P680 passa allo stato eccitato e traferisce un elettrone ad una
molecola di \textbf{feofitina (pheo)}. La feofitina è una molecola di
clorofilla in cui il Mg centrale è stato sostituito da due atomi di H, e
che agisce nel PSII come un accettore precoce. Tutto questo avviene in
tempi molto brevi, all'incirca 10-12 picosecondi.

In questo modo le cariche resterebbero separate per qualche nanosecondo,
dopodichè l'elettrone passato alla feofitina tornerebbe allo stato
iniziale. Da un punto di vista della fotosintesi però è necessario che
l'elettrone non ritorni indietro, ma che venga allontanato sia da un
punto di vista spazioale, che da da un punto di vista energetico. Più
alta è la differenza di energia infatti, più è improbabile che
l'elettrone torni indietro (ricombini).

La feofitina passa gli elettroni ad un complesso di due plastochinoni in
prossimità di un atomo di ferro. I due \textbf{plastochinoni PQ$_A$ e
PQ$_B$} sono legati al centro di reazione e ricevono elettroni dalla
feofitina in modo sequenziale.

Un primo elettrone viene trasferito al PQ$_A$ il quale è legato alla
proteina D2. I plastochinoni di solito accettano due riducenti
equivalenti, passando per una forma intermedia chiamata
\textbf{semichinone} (una parte ridotta e una ossidata). Il PQ$_A$ si
comporta dunque come un cofattore che trasferisce un elettrone.

Il PQ$_A$ si ``sbarazza'' del primo elettrone ricevuto si sbarazza prima
di accettare il secondo (quindi trasferisce un solo elettrone anche se
ne riceve due).

In questo modo il potenziale energetico dell'elettrone si abbassa poichè
il chinone ha un potenziale energetico minore della feofitina, e inoltre
si trova anche lontano spazialmente, poichè il PQ$_A$ si trova sulla
proteina D2. A questo punto la ricombinazione tra PQ$_A$$^-$ e P680$^+$
diventa molto meno probabile rispetto a quella tra feofitina$^-$ e
P680$^+$, e tutto questo avviene in microsecondi (10-6).

P680 ha ancora una carica positiva, ed ha un potenziale redox talmente
basso da poter ossidare qualsiasi molecola (se questa lacuna elettronica
non viene riempita il P680 provvede a ricavare elettroni da altre
parti). L'elettrone fisiologico che riempie il P680$^+$ arriva dalla
tirosina (Y), la quale è una molecola attiva da un punto di vista redox
(richiede all'incirca un millisecondo).

PQ$_A$ a questo punto si ``sbarazza'' dell'elettrone e lo trasferisce a
\textbf{PQ$_B$}, che ha un sito di legame apposito in una tasca della
proteina D1 (PQ$_B$ riceve elettroni solo quando è nella tasca della
proteina).

Il donatore di elettroni alla molecola di tirosina è un \emph{sistema
(clust) formato da 4 atomi di manganese e uno ione calcio}. Il manganese
è un elemento di transizione che può avere uno stato di ossidazione 2+ o
3+, e quando un manganese del sistema dona elettroni alla tirosina passa
dallo stato di ossidazione 2+ a quello 3+.

Una volta che due fotoni hanno colpito il P680 e due manganesi sono
passati dallo stato di ossidazione 2+ a 3+, il PQ$_B$ diventa
\textbf{PQB$^2$$^-$}. A questo punto PQB$^2$$^-$ acquista due protoni
(H$^+$) dalla parte stromatica della membrana, perde la sua affinità per
il sito di legame, e si dissocia dal complesso del centro di reazione
sotto la forma di \textbf{plastoidrochinone (PQH$_2$)}, o
\textbf{chinolo} completamente ridotto.

\textbf{IMMAGINE 7.26 p254}

Il plastoidrochinone a differenza dei grandi complessi proteici della
membrana tilacoidale, è una piccola molecola non polare che diffonde
rapidamente nel cuore non polare del bistrato membranoso; una volta
dissociatosi dal centro di reazione infatti, entra nella porzione
idrocarburica della membrana dove trasferisce i suoi elettroni al
\emph{complesso del citocromo b$_6$-f}.

Il sito di legame per PQ$_B$, dopo aver rilasciato PQH$_2$, resta vuoto
e un altro PQ$_B$ andrà ad occupare quella posizione.

Lo stesso processo si ripete per un terzo e quarto fotone. Il sistema di
manganese a questo punto ha accumulato 4 cariche positive e tutti i suoi
atomi di Mg hanno n.o. 3+.

Una molecola di ossigeno viene liberata ogni 4 eventi fotochimici, al
termine del processo di scambio degli elettronei (per ogni molecola
d'acqua vengono invece prodotti due chinoli). Il processo dura circa 15
millisecondi.

L'acqua viene ossidata a ossigeno e gli elettroni prodotti vengono
trasportati fino a trovarli sotto forma di chinoli. L'energia dei fotoni
è conservata sotto forma di energia di potenziale redox nei
plastochinoni.

Le proteine D1 e D2 non esistono nella membrana come tali ma esistono
due \emph{proteine antenna} chiamate \textbf{CP47} e \textbf{CP43} (CP
sta per ``proteina che lega la clorofilla'', il numero invece ha a che
fare con il peso molecolare).

Vi sono anche le \textbf{light harvesting complex II (LHCII)} che
catturano la luce per il fotosistema II. Qui ci sono la stragrande
maggioranza di molecole di clorofilla perchè sono i pigmenti antenna. I
geni che codificano per le proteine LHCII sono chiamati \textbf{LHCB}.

Mentre le proteine CP sono prodotte da geni contenuti nel cloroplasto, i
geni per le LHCII si trovano nel nucleo.

Nei \textbf{citocromi di tipo b} gli eme non sono legati covalentemente
alla proteina.

L'unità base della fotosintesi ossigenica è D1, D2, CP47 e CP43 in tutti
gli organismi fotosisntetici.

Quello che cambia dai cianobatteri alle piante superiori è l'LHCII,
ovvero il sistema che ha il ruolo di raccogliere la luce.

Dal punto di vista evolutivo i centri di reazione sono stati altamente
conservati, mentre i sistemi antenna si sono molto diversificati.

Ci sono altre proteine: le O, P e Q sono di più recente scoperta e sono
tre proteine adese alla membrana, estrinseche.

La \textbf{proteina O} ha a che fare con il cluster del manganese, la
\textbf{proteina P} contiene i siti di legame ionici per il cloro,
mentre della \textbf{proteina Q} non si conosce la funzione, tuttavia ha
una sequenza altamente conservata e quindi deve averne sicuramente una.

Nella parte finale della respirazione si trovano un chinolo ed un
ossigeno che, tramite l'\textbf{enzima citocromo ossidasi (COX)} vengono
trasformati in un chinone, 2 molecole di acqua ed energia. Il COX
preleva gli elettroni dall'ubichinolo e li trasferisce all'ossigeno.

Da un punto di vista termodinamico il PSII e la citocromo ossidasi
compiono la stessa reazione nelle due direzioni, ma la citocromo
ossidasi è termodinamicamente favorita mentre il fotosistema II è
sfavorito.

Tra il fotosistema I e il II si trova il \textbf{complesso del citocromo
b$_6$-f} (nei mitocondri si chiama \emph{bc1}). La f viene da ``folia''
e il tipo di citocromo è c. I citocromi hanno gruppi prostetici eme e
trasportano solo elettroni.

\paragraph{Il flusso di elettroni attraverso il complesso citocromo
b$_6$f}\label{il-flusso-di-elettroni-attraverso-il-complesso-citocromo-bux5f6f}

Il complesso del citocromo b$_6$f è una grande proteina costituita da
molte subunità e possiede numerosi gruppi prostetici. Contiene:

\begin{itemize}
\itemsep1pt\parskip0pt\parsep0pt
\item
  2 \textbf{gruppi eme} di tipo \emph{b} e 2 di tipo \emph{c}
  (\emph{citocromo f}). Nei citocromi di tipo \emph{c} l'eme è legato al
  peptide covalentemente, mentre nei citocromi di tipo \emph{b} il
  gruppo protoemico chimicamente simile non è legato covalentemente;
\item
  una \textbf{Fe-S proteina} di Rieske nella quale due atomi di Fe sono
  collegati tramite due atomi di S;
\item
  cofattori addizionali (un gruppo eme aggiuntivo, una clorofilla, un
  carotenoide di cui non si conosce ancora la funzione\ldots{}).
\end{itemize}

\paragraph{Il ciclo Q}\label{il-ciclo-q}

Il \textbf{citocromo bc1} dei mitocondri è un complesso proteico in cui
troviamo il \textbf{citocromo b}, il \textbf{citocromo c} ed un
\textbf{proteina Fe-S}. L'insieme di questo complesso proteico ha la
funzione di trasferire elettroni da un chinone ad una proteina che
trasporta solo elettroni (nei cloroplasti è la plastocianina a ricoprire
questo ruolo, mentre nei mitocondri è il citocromo c).

Nelle piante il ciclo degli elettroni si svolge tra PSII, complesso del
citocromo e PSI.

I complessi b$_6$f svolgono il \textbf{ciclo Q}.

In questo meccanismo il plastoidrochinone (PQH$_2$, detto anche
plastochinolo) è ossidato e uno dei due elettroni è trasferito tramite
una catena lineare di trasporto di elettroni al PSI, mentre l'altro
elettrone va incontro ad un processo ciclico che aumenta il numeor di
protoni pompati attraverso una la membrana.

Il PSII produce un chinolo che si lega al citocromo b$_6$f. Il chinolo
catturato viene ossidato a chinone liberando due elettroni e due ioni
H$^+$ che vengono immediatamente liberati nel lumen del cloroplasto;
all'aumentare di questa concentrazione si ha una diminuzione del pH
tramite un processo di acidificazione (il pH all'interno del lumen può
anche raggiungere un pH di 5,5 a differenza del pH dello stroma che
resta intorno a 8).

I \textbf{2 e$^-$} liberati seguono dunque due destini diversi.

Nella catena lineare di trasporto di elettroni lineare, la proteina
ossidata di Rieske (FeSR) accetta un elettrone dal PQH$_2$. Nei
\textbf{cluster Fe-S} lo zolfo ha n.o. 2- mentre il ferro inizialmente
si trova allo stato 3+, ma dopo aver acccettato l'elettrone diventa +2
(viene ridotto).

Successivamente l'elettrone viene trasferito al \emph{citocromo f} che a
sua volta lo trasferisce ad una molecola di \textbf{plastocianina}.
Questa via è definita ``in caduta di potenziale'', è spontanea.

La plastocianina, a sua volta, riduce il P700 ossidato del PSI.

Nella parte ciclica del processo il \textbf{plastosemichinone}
trasferisce il suo secondo elettrone a uno degli emi di tipo \emph{b} (i
citocromi \emph{b} possiedono \textbf{due gruppi eme}, uno a basso
potenziale e uno ad alto potenziale, che si trovano sempre nel
complesso), rilasciando i suoi due protoni nella parte del lume della
membrana. Il primo eme \emph{b} trasferisce il proprio elettrone,
attraverso il secondo eme \emph{b}, a una molecola ossidata di
plastochinone, riducendolo a semichinone vicino alla superficie del
complesso vicina allo stroma.

Un'altra sequenza simile di flusso di elettroni riduce completamente il
plastochinone, che cattura protoni dalla faccia stromatica della
membrana e viene liberato dal complesso \emph{b$_6$f} come
plastoidrochinone.

\textbf{IMMAGINE 7.28 p257}

Il risultato complessivo dei due cicli è che due elettroni vengono
trasferiti al P700, due plastoidrochinoni vengono ossidati a
plastochinoni e un plastochinone viene ridotto a plastoidrochinone. Nel
processo di ossidaizone dei plastochinoni vengono trasferiti 4 protoni
dalla faccia stromatica a quella del lume della membrana. In questo modo
il flusso di elettroni che collega la parte accettore del centro di
reazione del PSII alla parte donatore del centro di reazione del PSI
genera anche un potenziale elettrochimico attraverso la membrana. Questo
potenziale è utilizzato per la sintesi di ATP.

Questo flusso ciclico attraverso il citocromo \emph{b} e il
plastochinone aumenta il numero di protoni pompati per ogni elettroni.
Questo complesso costituisce lo \textbf{step elettrogenico} della
fotosintesi.

\paragraph{Trasporto degli elettroni tra il PSII e il
PSI}\label{trasporto-degli-elettroni-tra-il-psii-e-il-psi}

La disposizione dei due fotosistemi in parti differenti delle membrane
tilacoidali necessita, per poter cedere gli elettroni prodotti dal PSII
al PSI, che almeno un componente sia in grado di muoversi nella
membrana. Il complesso \emph{b$_6$f} è troppo grande per poter agire
come trasportatore mobile.

Questo compito è svolto dal plastochinone e dalla
\textbf{plastocianina}.

\textbf{IMMAGINE 7.22 p249}

La plastocianina (PC) è una piccola proteina idrosolubile contenente un
atomo di rame (per questo ha un colore azzurro) in grado di trasferire
elettroni fra il complesso del citocromo *b$_6$f e il P700. Questa
proteina è localizzata all'interno del lume.

La plastocianina ossidata ha rame rameico 2+, mentre quella ridotta ha
rame rameoso 1+.

Il complesso del citocromo \emph{b$_6$f} catalizza la reazione, ossia il
trasferimento di elettroni dal chinolo alla plastocianina. Il chinolo
contiene elettroni e ioni H$^+$ mentre la plastocianina trasporta solo
elettroni. Il \emph{b$_6$f} da una parte trasferisce elettroni alla
plastocianina e dall'altra libera gli ioni H$^+$ del chinolo all'interno
del lume del cloroplasto che in questo modo diventa acido

La reazione è:

\textbf{PQH$_2$ (plasochinolo) + Pc (plastocianina ossidata)
$\leftrightarrow$ PQ (plastochinone) + Pc (ridotta)}

È una reazione termodinamicamente favorita.

I geni delle proteine del complesso vengono chiamati \textbf{PET
(protein electron transfer)}. Le PET A, B e D sono codificate da geni
plastidiali, mentre le altre sono codificate da geni nucleari. Il
citocromo f nella forma ridotta 2+ può donare elettroni alla
plastocianina che varia tra lo stato di ossidazione 2+ e 1+. La
plastocianina si trova libera nel lumen del cloroplasto e ``nuota'' nel
lumen finchè non viene catturata dal PSI che possiede un sito di legame
per la PC.

\paragraph{Il centro di reazione del PSI e la riduzione del
NADP$^+$}\label{il-centro-di-reazione-del-psi-e-la-riduzione-del-nadp}

Il centro di reazione del PSI è un grande complesso formato da molte
subunità. La parte integrale del PSI è costituita da un nucleo formato
da circa 100 molecole di clorofilla (a differenza del PSII dove le
clorofille antenna sono associate al centro di reazione ma presenti in
pigmento-proteine distinte). Questo centro antenna e il P700 sono legati
a due proteine, \textbf{PsA} e \textbf{PsB} (formano un eterodimero). I
pigmenti antenna del nucleo formano una sfera che circonda i cofattori
di trasferimento elettronico posti al centro del complesso.

I carriers di elettroni che agiscono nella regione accettore del PSI
sono, nella loro forma ridotta, degli agenti estremamente riducenti. Tra
queste specie ridotte (e molto instabili) troviamo come accettori di
elettroni una molecola di clorofilla detta \textbf{A$_0$}. Questa
trasferisce elettroni ad un altro accettore, un fillochinone, chiamato
\textbf{A1} e poi ad un \textbf{centro Fe-S}. Il centro Fe-S contiene 3
Fe-S proteine conosciute come \textbf{FeS$_X$}, \textbf{FeS$_A$} e
\textbf{FeS$_B$}. Gli elettroni sono trasferiti attraverso i centri A e
B alla \textbf{ferredossina (Fd)}.

\textbf{IMMAGINE 7.29 p258}

La ferrodossina è il riducente cellulare più forte che esista.

La flavoproteina associata alla membrana \textbf{ferredossina-NADP
reduttasi (FNR)} riduce il NADP$^+$ a NADPH, completando così la
sequenza di trasporto non ciclico di elettorni iniziata con
l'ossidazione dell'acqua.

Gli elettroni passano dall'acqua alla ferredossina con una differenza di
potenziale complessiva di 1140 mV.

Una determinata quantità di complesso del citocromo \emph{b$_6$f} si può
trovare nella regione stromatica della membrana dove è localizzato il
PSI. In determinate condizioni, può avvenire un \textbf{flusso ciclico
di elettroni} (detto anche \textbf{fosforilazione fotosintetica}) dalla
porzione ridotta del PSI, passando per il complesso *b$_6$f\$, per
ritornare poi al P700.

Questo flusso ciclico di elettroni è accoppiato al pompaggio di protoni
nel lume, che può essere utilizzato per la sintesi di ATP ma che non è
in grado di ossidare l'acqua o di ridurre il NADP$^+$. L'energia della
luce in questo ciclo viene conservata solo sotto forma di un $\Delta$pH.

La \textbf{NADPH deidrogenasi} potrebbe essere il complesso chiave per
la fosforilazione fotosintetica ciclica; nel genoma dei cloroplasti sono
state individuate sequenze geniche che ricordano quelle della NADH
deidrogenasi. Questi complessi esistono e potrebbero essere
potenzialmente usati.

La differenza di concentrazione degli ioni H$^+$ costituisce la forza
motrice per la formazione di ATP.

Ci troviamo in condizioni standard a pH 7, temperatura di 298,16 K e
pressione di 1atm. Il $\Delta$G di una reazione è uguale a zero quando
la costante di equilibrio è 1 (ovvero $\Delta$G è 0 quando la reazione è
all'equilibrio). Quasi sempre nei sistemi biologici però non ci troviamo
in situazioni di equilibrio; man mano che ci si allontana
dall'equilibrio l'energia necessaria per fare ATP può variare (sia
aumentare che diminuire) in funzione delle concentrazioni reali nella
cellula di ATP, ADP e P$_i$.

Una condizione tipica dei cloroplasti di una pianta che sta crescendo in
condizioni ottimali è che la quantità di energia necessaria per produrre
ATP è di circa -47,8 KJ/mole.

Nei nostri eritrociti invece, questo valore è di circa -53 KJ/mole.
Questo valore dipende dunque dal sistema biologico in cui ci troviamo.

\subsection{Trasporto di protoni e sintesi di ATP nel
cloroplasto}\label{trasporto-di-protoni-e-sintesi-di-atp-nel-cloroplasto}

\textbf{Lezione 20151027}

Per convenzione la parte della membrana da dove provengono gli ioni
H$^+$ è definita \textbf{``in''}, mentre quella dove vanno a finire gli
ioni è definita \textbf{``out''}. Oltre a generarsi un gradiente di
concentrazione si genera anche un potenziale elettrico (si parla di
\textbf{chemiosmosi}).

All'interno dei cloroplasti il $\Delta$pH è una condizione realistica:
un $\Delta$pH 3 indica che vi è una differenza di concentrazione ionica,
tra dentro e fuori, di 1000 volte. La quantità di energia che si libera
dipende dalla differenza di concentrazione: più alta è la differenza di
concentrazione, più alta è l'energia che si libera.

In condizioni realistiche, quando uno ione H$^+$ passa attraverso la
membrana, si libera una quantità di energia pari a circa -17 KJ/mol.
Questo significa che, se per una mole di ATP ho bisogno di almeno -47,8
KJ, allora devono passare almeno 3 ioni H$^+$ per produrla.

Una parte dell'energia luminosa catturata viene utilizzata per la
sintesi di ATP con un processo noto come \textbf{fotofosforilazione}. La
fotofosorilazione è resa possibile dal \emph{meccanismo chemiosmotico};
il principio su cui si basa la chemiosmosi è che le differenze di
concentrazione degli ioni e del potenziale elettrico attraverso la
membrana rappresentino delle fonti di energia libera che può essere
utilizzata dalla cellula.

L'ATP è sintetizzata da un complesso enzimatico noto con il nome di
\textbf{ATPsintati} o \textbf{ATPasi}. Questo enzima è formato da due
parti:

\begin{enumerate}
\def\labelenumi{\arabic{enumi}.}
\itemsep1pt\parskip0pt\parsep0pt
\item
  una porzione idrofoba legata alla membrana definita \textbf{CF$_0$};
\item
  una porzione che si protrude verso lo stroma definita \textbf{CF$_1$}.
\end{enumerate}

Si è proposto che durante la catalisi una grande porzione del complesso
CF$_1$ ruoti su un monoalbero formato dalla subunità $\gamma$. CF$_1$ è
composto da diversi peptidi, comprese 3 copie dei polipeptidi $\alpha$ e
$\beta$, disposti in modo alternato. I siti catalitici sono
principalmente localizzati sui polipeptidi $\beta$. Questa è la porzione
del complesso che sintetizza ATP. La subunità $\gamma$ agisce ruotando
alternativamente contro le subunità $\alpha$ e $\beta$. L'energia dei
movimenti conformazionali è quindi tradotta in energia di legame
fosfoanidride.

Il CF$_0$ ha la funzione di canale transmembrana attraverso il quale
vengono traslocati i protoni dal lume del cloroplasto alla parte
catalitica dell'enzima. La porzione CF$_1$ invece catalizza la
conversione di ADP e fosfato inorganico in ATP.

La subunità CF$_0$ funziona come una sorta di motorino che girando
produce cambi conformazionali nelle subunità $\alpha$ e $\beta$.

Le subunità della porzione CF$_1$ possono assumere 3 diverse
conformazioni: L (light), T (tight) e O (open).

Si ritiene che l'ADP e il P$_i$ si leghino prima al sito O per muoversi
poi verso il sito L. Il sito T lega l'ATP. Queste 3 diverse
conformazioni possiedono dunque diverse affinità per i substrati: nel
sito O l'ADP e l'ATP possono scambiarsi abbastanza rapidamente perché
questa conformazione ha bassa affinità per i substrati. La conformazione
L lega ADP e P$_i$, mentre la conformazione T è quella catalitica in cui
viene fromato l'ATP.

\textbf{IMMAGINE S7.9.A p263}

Dal sito attivo della conformazione T vengono espulse delle molecole di
acqua; l'espulsione di queste molecole rende la sintesi di ATP un
processo spontaneo. \textbf{VEDERE QUADERNO JESS}

L'energia è liberata come i protoni si spostano attraverso il canale
CF$_0$ passando dal lume alla regione stromatica, e la subunità $\gamma$
del CF$_1$ ruota. La rotazione causa cambiamenti conformazionali nei tre
siti di legame per il nucleotide, facendoli interconvertire e cambiando
così l'affinità dei siti per i nucleotidi. L'ATP è liberata quando il
sito T si converte nel sito O, dando il via ad un altro ciclo di
reazioni.

Esistono anche delle \textbf{V-ATPasi}, ovvero delle ATPasi vacuolari la
cui attività catalitica consiste nell'idrolisi dell'ATP. Le
\textbf{P-ATPasi} invece si trovano sulle membrane delle cellule che
fungono da pompe ioniche. Queste sono pompe per la concentrazione degli
ioni H$^+$ e non per quella Na$^+$/K$^+$ (il sodio nelle piante non
c'è).

\subsection{La fase buio della
fotosintesi}\label{la-fase-buio-della-fotosintesi}

Tramite la fase luminosa della fotosintesi le cellule immagazzinano
energia sotto ATP e NADPH. Queste reazioni sono mediate esclusivamente
dalle proteine presenti nelle membrane tilacoidali.

La seconda parte della fotosintesi riguarda invece il modo in cui queste
due molecole vengono utilizzate, ed uno dei processi in cui sono
implicate è quello dell'\textbf{assimilazione del carbonio}.

La concentrazione della CO$_2$ nell'aria è circa dello 0.03\% (ad oggi
circa dello 0.034\%). Per ogni litro di aria c'è qualche micromole di
CO$_2$ (3mM). Nell'aria c'è 600 volte più ossigeno che CO$_2$.

La fase oscura è stata studiata da Calvin e Benson.

Per studiare il destino della CO$_2$ si utilizzò un isotopo del carbonio
radioattivo per la marcatura. Il carbonio organico è più ridotto
rispetto a quello presente nella CO$_2$. L'assimilazione del carbonio,
dal punto di vista chimico è un processo di riduzione.

L'esperimento era basato sull'utilizzo di:

\begin{itemize}
\itemsep1pt\parskip0pt\parsep0pt
\item
  una vasca contenete l'alga verde unicellulare \emph{Chlorella};
\item
  una lampada;
\item
  un filtro termico;
\item
  una beuta contenente etanolo bollente.
\end{itemize}

I ricercatori esposero le cellule dapprima a condizioni costanti di luce
e CO$_2$, in modo tale da stabilire lo stato stazionario della
fotosintesi. Quindi fu aggiunta per un breve periodo della
$^1$$^4$CO$_2$ radioattiva così da marcare i diversi composti intermedi
del ciclo. Le cellule furono uccise tramite l'immersione nell'etanolo
bollente. I composti marcati $^1$$^4$C furono separati e identificati
tramite una \textbf{cromatografia bidimensionale su carta}.

La cromatografia su carta viene effettuata utilizzando una goccia della
soluzione che vogliamo analizzare su un pezzo di carta specifica,
dopodichè si intinge un bordo del foglio in un solvente specifico. Il
solvente imbeve la carta e, risalendo lungo questa, arriva alla goccia
di soluzione; le diverse molecole che compongono la nostra soluzione
hanno diverse affinità per il solvente e dunque tenderanno a
suddividersi in ``bande'' in base a quanto e a quale velocità il
solvente è in grado di spostarle. Dopo questo primo passaggio si può
anche girare il lato del foglio di 180° e immergerlo in un secondo
solvente. In questo modo conosceremo l'affinità delle molecole per due
tipi di solventi diversi.

In seguito alla cromatografia, poichè le alghe erano state coltivate in
presenza di $^1$$^4$CO$_2$, la cromatografia fu posizionata su una
lastra che rimase impressa grazie alle radiazioni emesse dalle molecole
che avevano organicato la CO$_2$ radioattiva. Tramite quest'esperimento
fu dunque possibile individuare quali e quante molecole diverse venivano
formate durante la fotosintesi.

In questo modo si scoprì che la radioattività, se la cellula veniva
esposta a CO$_2$ radioattiva per tempi brevi, veniva accumulata tutta
nel \textbf{3-fosfoglicerato}, mentre in seguito a esposizioni più
lunghe si otteneva una moltitudine di composti (soprattutto amminoacidi,
zuccheri e acidi organici). Da questo si dedusse che il 3-fosfoglicerato
era il precursore di una moltitudine di molecole differenti.

Facendo la degradazione chimica di queste molecole Calvin scoprì che la
radiottività era localizzata sul C1 sotto forma di gruppo carbossilico.
La prima idea fu che la CO$_2$ venisse incorporata su un accettore a due
atomi di carbonio, ma questo ipotetico accettore non venne mai trovato.
Le molecole analizzate dopo tempi di esposizione più lunghi,
presentavano il $^1$$^4$C non solo sul primo gruppo: questo fece
supporre che vi fosse un rimescolamento degli atomi ci C. L'esperimento
chiave che fece comprendere il reale funzionamento di questo meccanismo
fu il fatto che venne osservato che in condizioni di \emph{bassa CO$_2$}
il \textbf{ribulosio 1,5-bisfosfato} si accumulava, mentre in condizioni
di \emph{alta CO$_2$} il ribulosio 1,5-BisP tendeva a scomparire.

Questo fece supporre che fosse proprio il ribulosio 1,5-BisP (molecola a
5 atomi di C) a reagire con reagire con la CO$_2$, e non una molecola
2C.

Il ribulosio 1.5-Bisfosfato ha una struttura ciclica. Il carbonio basico
del ribulosio e quello acido della CO$_2$ possono reagire in una
reazione acido-base secondo Lewis e formare un composto ramificato a 6
atomi di C. In presenza di acqua il composto ramificato è instabile e
spontaneamente si idrolizza in due molecole di \emph{fosfoglicerato}
isomericamente identiche.

L'enzima che catalizza questa reazione è l'enzima \textbf{rubisco}
(catalizza anche una reazione di ossigenazione). Questo enzima è
caratterizzato da una velocità massima di reazione estremamente bassa
(3-4 turnover al secondo). Questo enzima presenta un'elevata affinità
per il suo substrato e dunque, nonostante la CO$_2$ sia presente in
atmosfera a basse concentrazioni, l'enzima funziona al massimo delle sue
possibilità.

La rubisco è composta da due tipi di subunità: una più grande chiamata
\textbf{L} ed una più piccola chiamata \textbf{S}. Questa proteina è
codificata in parte da un gene contenuto in un'unità plastidiale, mentre
in parte è codificata da geni nucleari. Il suo assemblaggio è quindi
molto complesso. Ciascuna molecola di rubisco è composta da 8 subunità L
catalitiche e altre 8 subunità S di cui non si conosce la funzione. Le
subunità sono organizzate come tetrameri, ce ne sono 2 una sopra e una
sotto a formare un tappo. I siti attivi dell'enzima sono rappresentati
da delle lisine (hanno la capacità di favorire la tautomerizzazione del
ribulosio).

(\textbf{29/11}) Modifica ordine argomenti

Il ciclo di Calvin-Benson si compie in 3 fasi che sono altamente
coordinate nei cloroplasti:

\begin{enumerate}
\def\labelenumi{\arabic{enumi}.}
\itemsep1pt\parskip0pt\parsep0pt
\item
  \textbf{Carbossilazione} della molecola accettore di CO$_2$ per
  generare due molecole di un composto intermedio a 3 atomi di C
  (3-fosfoglicerato);
\item
  \textbf{Riduzione} del 3-fosfoglicerato a carboidrati a 3 atomi di C;
\item
  \textbf{Rigenerazione} dell'accettore di CO$_2$, il ribulosio
  1,5-Bisfosfato.
\end{enumerate}

\textbf{IMMAGINE 8.2 p279}

\subsubsection{La carbossilazione}\label{la-carbossilazione}

\begin{itemize}
\itemsep1pt\parskip0pt\parsep0pt
\item
  \textbf{Reazione 1}
\end{itemize}

Nella prima fase del ciclo 3 molecole di CO$_2$ e 3 di H$_2$O reagiscono
con 3 molecole di ribulosio 1,5-bisfosfato per formare 6 molecole di
\textbf{3-fosfoglicerato}. Questa reazione è catalizzata dall'enzima del
cloroplasto ribulosio-1,5-bisfosfato carbossilasi/ossigenasi
\textbf{(rubisco)}

\textbf{3 rubisco + 3 CO$_2$ + 3 H$_2$O $\leftrightarrow$ 6
3-fosfoglicerato + 6 H$^+$}

3CO2 + 3 Ru 1,5B $\leftrightarrow$ 6 PGA 3 C + 15 C (= 18)
$\leftrightarrow$ 6 (x3) (= 18)

Nella prima reazione parziale, un H$^+$ è estratto dal carbonio in 3 del
ribulosio 1,5-BisP.

Questo origina la forma reattiva dello zucchero, ovveor quella
\emph{enediolica} (-diolica perché legati ai 2 atomi di C impegnati nel
doppio legame ci sono 2 atomo di O).

Il doppio legame di questa molecola ha una carica negatica perché è
ricco di elettroni (sono carboni basici, mentre quello della CO$_2$ è un
carbonio acido perchè povero di elettroni).

L'aggiunta di CO$_2$ gassosa all'intermedio instabile enediolo legato
alla rubisco, conduce alla seconda reazione parziale per la formazione
\emph{irreversibile} di un intermedio a 6 atomi di C ramificato
(2-carbossi-3-chetoarabinitol 1,5-bisfosfato). Questo è un composto
instabile.

L'idratazione di questo intermedio porta a 2 molecole,
stereoisomericamente identiche, di 3-fosfoglicerato.

Questa reazione è importante perché il carbonio inorganico può diventare
carbonio organico (questa è l'unica via di ingresso del C organico nella
biosfera).

\subsubsection{La riduzione}\label{la-riduzione}

Questa fase riduce il carbonio del 3-fosfoglicerato proveniente dallo
stadio di carbossilazione grazie all'utilizzo di ATP e NADPH generati
durante le reazioni alla luce della fotofosforilazione.

Questa fase si svolge in due reazioni:

\begin{itemize}
\itemsep1pt\parskip0pt\parsep0pt
\item
  \textbf{Reazione 2}
\end{itemize}

L'ATP formato nelle reazioni alla luce fosforila il 3-fosfoglicerato al
gruppo carbossilico producendo un'\emph{anidride mista},
l'\textbf{1,3-bisfosfoglicerato}, in una reazione catalizzata dalla
\textbf{3-fosfoglicerato chinasi}.

\begin{itemize}
\itemsep1pt\parskip0pt\parsep0pt
\item
  \textbf{Reazione 3}
\end{itemize}

Successivamente il NADPH riduce l'1,3-bisfosfoglicerato a
\textbf{gliceraldeide 3-fosfato}, in una reazione catalizzata dalla
\textbf{NADP-gliceraldeide-3-fosfato deidrogenasi}.

\subsubsection{La rigenerazione}\label{la-rigenerazione}

Per evitare l'esaurimento delle sostanze intermedie del ciclo di
Calvin-Benson, la continua assimilazione di CO$_2$ atmosferica richiede
la rigenerazine costante dell'accettore della CO$_2$, il ribulosio
1,5-bisfosfato.

Nella fase di rigenerazione 3 molecole di rib 1,5-BisP si formano dalle
reazioni che rimescolano i C di 5 molecole di Glic 3P. La sesta molecola
di Glic 3P rappresenta l'assimilazione al netto delle 3 molecole di
CO$_2$ e diventa disponibile per il metabolismo del carbonio della
pianta.

\begin{itemize}
\itemsep1pt\parskip0pt\parsep0pt
\item
  \textbf{Reazione 4}
\end{itemize}

\emph{Due molecole} di gliceraldeide 3-fosfato vengono convertite
tramite isomerizzazione in \textbf{diidrossiacetone fosfato (DHAP)}, un
chetone, nella reazione catalizzata dalla \textbf{trioso fosfato
isomerasi}.

\begin{itemize}
\itemsep1pt\parskip0pt\parsep0pt
\item
  \textbf{Reazione 5}
\end{itemize}

Una molecola di DHAP subisce una condensazione aldolica con una
\emph{terza molecola} di gliceraldeide 3P (DHAP + Glic3P), una reazione
catalizzata dalla \textbf{aldolasi}, per formare \textbf{fruttosio
1,6-bisfosfato}.

Queste prime due reazioni sono \emph{reversibili}.

\begin{itemize}
\itemsep1pt\parskip0pt\parsep0pt
\item
  \textbf{Reazione 6}
\end{itemize}

Il fruttosio 1,6-bisP è idrolizzato a \textbf{fruttosio 6-fosfato} in
una reazione catalizzata da una \textbf{fruttosio 1,6-bisfosfatasi}
specifica del cloroplasto.

Questa è una reazione irreversibile poichè per tornare indietro
occorrerebbe fornire energia (c'è perdita di un fosfato). Questo è un
punto di controllo del ciclo.

\begin{itemize}
\itemsep1pt\parskip0pt\parsep0pt
\item
  \textbf{Reazione 7}
\end{itemize}

Un'unità a 2 atomi di C della molecola del fruttosio 6P (C1 e 2) viene
trasferita attraverso l'enzima \textbf{transchetolasi} (trasporta gruppi
chetilici) ad una \emph{quarta molecola} di gliceraldeide 3P per formare
\textbf{xilulosio 5-fosfato}. Gli altri 4 atomi di C del fruttosio
formano l'\textbf{eritrosio 4-fosfato} Lo xilulosio 5P viene
isomerizzato a \textbf{ribulosio 5P}.

\begin{itemize}
\itemsep1pt\parskip0pt\parsep0pt
\item
  \textbf{Reazione 8}
\end{itemize}

L'eritrosio 4P si unisce poi, tramite l'aldolasi (condensazione
aldolica), con la restante molecola di DHAP per formare lo zucchero a 7
atomi di C, il \textbf{sedoeptulosio 1,7-bisfosfato}.

\begin{itemize}
\itemsep1pt\parskip0pt\parsep0pt
\item
  \textbf{Reazione 9}
\end{itemize}

Il sedoeptulosio 1,7-BisP viene poi idrolizzato e defosforilato per
mezzo di una fosfatasi specifica, la \textbf{sedoeptulosio
1,7-bisfosfatasi}, per formare \textbf{sedoeptulosio 7-fosfato}.

Questa è la seconda reazione irreversibile; l'enzima che catalizza
questa reazione è sottoposto a regolazione.

\begin{itemize}
\itemsep1pt\parskip0pt\parsep0pt
\item
  \textbf{Reazione 10}
\end{itemize}

Il sedoeptulosio 7P dona un'unità a 2 atomi di C (C1 e 2) alla
\emph{quinta molecola} di gliceraldeide 3P, tramite una
\textbf{transchetolasi}, producendo \textbf{xilulosio 5-fosfato}. I
rimanenti 5 atomi di C della molecola di sedoeptulosio 7P diventano
\textbf{ribosio 5-fosfato}.

\begin{itemize}
\itemsep1pt\parskip0pt\parsep0pt
\item
  \textbf{Reazione 11a}
\end{itemize}

Due molecole di xilulosio 5P vengono convertite in due molecole di
\textbf{ribulosio 5-fosfato} da una \textbf{ribulosio 5-fosfato
epimerasi}.

\begin{itemize}
\itemsep1pt\parskip0pt\parsep0pt
\item
  \textbf{Reazione 11b}
\end{itemize}

La terza molecola di ribulosio 5P proviene dal ribosio 5-fosfato per
azione della \textbf{ribosio 5-fosfato isomerasi}.

\begin{itemize}
\itemsep1pt\parskip0pt\parsep0pt
\item
  \textbf{Reazione 12}
\end{itemize}

Infine la \textbf{fosforibulochinasi} catalizza la fosforilazione di tre
molecole di ribulosio 5-fosfato con ATP, in modo da rigenerare le tre
molecole necessarie dell'accettore iniziale della CO$_2$, il ribulosio
1,5-bisfosfato.

\textbf{IMMAGINE 8.3 p281}

La resa di questa reazione è abbastanza alta (nel glucosio che si forma
troviamo il 33\% dell'energia ottenuta inizialmente dalla luce).

Per formare una molecola di glucosio ci voglio almeno 12 molecole di
NADPH e 18 di ATP.

Se la rubisco funzionasse sempre, poichè il ciclo di Calvin consuma ATP
e NADPH che di notte non sono abbondanti, si dovrebbe trovare un altro
processo per la produzione di queste molecole: la respirazione.
Tuttavia, se una pianta respira troppo consuma i prodotti della
fotosintesi, di notte quindi la rubisco deve essere ``spenta''.

Questo enzima, oltre ad una funzione carbossilasica (è una riduzione),
ha anche una funzione ossigenica (ovvero può ossidare le molecole).

\subsection{Regolazione del ciclo di
Calvin-Benson}\label{regolazione-del-ciclo-di-calvin-benson}

L'uso efficiente dell'energia nel ciclo di C-B presuppone l'esistenza di
specifici meccanismi di regolazione che garantiscono non solo che tutti
i prodotti intermedi del ciclo siano presenti in contrazioni adeguate
alla luce, ma anche che il ciclo sia spento quando non è necessaria la
sua funzione al buio.

Due meccanismi generali compiono la modifica delle proprietà cinetiche
degli enzimi:

\begin{itemize}
\itemsep1pt\parskip0pt\parsep0pt
\item
  cambiamenti nei legami covalenti che si traducono in un enzima
  modificato chimicamente;
\item
  modifiche di interazioni non covalenti causate dal legame con
  metaboliti o da variazioni nella composizione ionica del mezzo
  cellulare.
\end{itemize}

La molecola di CO$_2$ gioca un doppio ruolo nell'atività della rubisco:
la CO$_2$ partecipa alla trasformazione dell'enzima dalla forma inattiva
a quella attiva ed è il substrato per la reazione di carbossilazione.

Come \emph{modulatore} la CO$_2$ reagisce lentamente con il gruppo
amminico di una specifica lisina all'interno del sito attivo della
rubisco. Il carbammato risultante che ne deriva si lega rapidamente al
Mg$^2$$^+$ per rendere l'enzima attivo.

Oltre alla rubisco la luce controlla l'attività di quattro enzimi del
ciclo di Calvin-Benson attraverso il \textbf{sistema
ferredossina-tioredossina}.

Alla luce la ferrodossina contiene Fe$^2$$^+$ (ridotto) mentre al buio
contiene Fe$^3$$^+$ (ossidato); questo enzima può ridurre molte
molecole.

le \textbf{tioredossine} possiedono un \emph{gruppo bisolfuro}
cataliticamente attivo. Le tioredossine possiedono un sito attivo con
due residui di cisteina redox-attivi. Il sito attivo è localizzato sulla
prima $\alpha$ elica di un motivo formato da tre $\alpha$ eliche ed un
foglietto $\beta$ tetra-intrecciato.

La forma ossidata di ogni tioredossina contiene un ponte bisolfuro
(-S-S-) che è ridotto a sulfidrile (-SH) sia dalla ferredossina ridotta
sia dal NADPH. La forma ridotta della tioredossina è un catalita
eccellente per la riduzione di legami bisolfuro intramolecolari di
proteine che sono debolmente ridotte dal glutatione, l'altro principale
riducente di sulfidrili nella cellula.

A seguito della riduzione, il legame bisolfuro degli enzimi colpiti
specificatamente dalla tioredossina causa un cambiamento drammatico
nell'attività catalitica (di solito aumenta).

Nelle cellule fotosintetiche esisto 3 varianti di tioredossine. Le due
tioredossine del cloroplasto sono ridotte da elettroni provenienti dalla
clorofilla eccitata tramite la ferredossina e la
\textbf{ferredossina:tioredossina reduttasi (FTR)}, un ferro-zolfo
enzima con un gruppo bisolfuro cataliticamente attivo; questo enzima può
trasformare i ponti disolfuro in gruppi sulfidrilici.

\subsection{La fotorespirazione o ciclo
C$_2$}\label{la-fotorespirazione-o-ciclo-cux5f2}

La rubisco ha la capacità di catalizzare sia la carbossilazione che
l'ossigenazione del ribulosio 1,5-BisP.

L'ossigenazione ribulosio 1,5-BisP dà inizio ad una rete coordinata di
reazioni enzimatiche che sono compartimentate nei cloroplasti, nei
perossisomi (della foglia) e nei mitocondri. Questo processo è noto come
\textbf{fotorespirazione} e provoca la perdita parziale della CO$_2$
fissata dal ciclo di C-B e l'assimilazione di ossigeno.

L'impatto negativo sulla crescita delle piante di queste reazioni
concorrenti è stato dimostrato con varietà dimutanti di
\emph{Arabidopsis} per la fotorespirazione che mostrrano una crescita
ritardata alla normale concentrazione atmosferica di CO$_2$ ma sono
normali in ambienti con CO$_2$ elevata.

Come substrati alternativi per la rubisco, CO$_2$ e O$_2$ competono per
la reazione con il ribulosio 1,5-BisP perchè la carbossilazione e
l'ossigenazione si verificano all'interno dello stesso sito attivo.

Cos'è che determina il comportamento dell'enzima in una senso invece che
nell'altro? Il rapporto {[}CO$_2${]} / {[}O$_2${]}.

Se questo rapporto è pari a 1, la rubisco ha una tendenza a carbossilare
di 80 volte maggiore rispetto alla sua tendenza ad ossigenare, questo
perchè la rubisco è molto più affine alla CO$_2$ che non all'O$_2$.

Esistono però condizioni in cui, in presenza di un'elevata
concentrazione di O$_2$ rispetto alla concentrazione della CO$_2$, la
pianta può compiere anche solo l'ossigenazione (fotorespirazione).

La CO$_2$ che la rubisco ha a disposizione è quella che dall'atmosfera
si scioglie nei liquidi della pianta; quanta CO$_2$ diffonde nei liquidi
dipende dalla concentrazione di CO$_2$ nell'aria e dal calore del
liquido.

La solubilità è regolata da \textbf{S$_t$ = S$_0$ (1 + $\lambda$T)}
(CONTROLLARE)

Dove \emph{S} è la solubilità standard e \emph{$\lambda$} è la lunghezza
d'onda. Con l'aumentare della temperatura la solubilità dei gas
diminuisce.

Se la solubilità per O$_2$ e CO$_2$ è diversa e a una data temperatura
ho un certo rapporto tra i due elementi, ad un'altra temperatura il
rapporto sarà diverso per effetto di $\lambda$. Alle alte temperature la
solubilità dell'ossigeno varia meno rispetto a quella dell'anidride
carbonica e per questo motivo all'aumentare della temperatura aumenta
l'attività ossigenica.

Ciò non si verifica sempre perché le piante, come tutti gli esseri
viventi, hanno una grande capacità di adattamento.

Le soluzioni per contrastare questo fenomeno sono due: si può modificare
l'attività ossigenasica della rubisco, oppure si può costruire un
meccanismo istologico che metta la rubisco nelle condizioni di lavorare
in maniera ottimale (tipo formare pompe per l'accumulo della CO$_2$). La
seconda è la soluzione realmente attuata.

Durante l'evoluzione l'attività ossigenasica della rubisco non è stata
modificata molto, probabilmente a causa del fatto che questo enzima è
l'unico enzima presente nei vegetali a compiere questo tipo di reazione.

Esistono molti organismi che vivono in condizioni anossigeniche; questi
sono organismi che vivono ancora secondo le condizioni ambientali
anaerobiche primitive, e questo fa pensare che l'enzima rubisco, dopo la
formazione del PSII, si sia trovato a dover fare i conti con l'O$_2$, ma
a quel punto dell'evoluzione l' enzima era ormai troppo evoluto per
poter modificare la sua funzione senza creare grossi danni alla pianta.
Si è cercato, tramite esterimenti di mutagenesi, di modificare il sito
attivo della rubisco, ma sono stati ottenuti solo enzimi peggiori. La
funzione sembra una sorta di memoria evolutiva dell'enzima.

\begin{itemize}
\itemsep1pt\parskip0pt\parsep0pt
\item
  \textbf{Reazione 2.1}
\end{itemize}

L'incorporazione di una molecola di O$_2$ nell'isomero 2,3-enediolo del
ribulosio 1,5-BisP genera un intermedio instabile che si divide
rapidamente in \textbf{2-fosfoglicolato} e \textbf{3-fosfoglicerato}.

\begin{itemize}
\itemsep1pt\parskip0pt\parsep0pt
\item
  \textbf{Reazione 2.2}
\end{itemize}

Il \emph{2-fosfoglicolato} formato nei cloroplasti dall'ossigenazione
del ribulosio 1,5-BisP viene rapidamente idrolizzato a
\textbf{glicolato} da una fosfatasi specifica dei cloroplasti.

Il successivo metabolismo dei glicolato prevede la cooperazione di due
altri organuli: mitocondri e perossisomi.

\begin{itemize}
\itemsep1pt\parskip0pt\parsep0pt
\item
  \textbf{Reazione 2.3}
\end{itemize}

Il glicolato esce dai cloroplasti attraverso una proteina trasportatrice
specifica della membrana interna e migra verso i perossisomi. Qui la
\textbf{glicolato ossidasi}, in presenza di cofattori *FADH$^+$,
catalizza l'ossidazione del glicolato producendo \textbf{H$_2$O$_2$} e
\textbf{gliossilato} e **FADH$_2$.

\begin{itemize}
\itemsep1pt\parskip0pt\parsep0pt
\item
  \textbf{Reazione 2.4 e 2.5}
\end{itemize}

La catalisi degrada l'H$_2$O$_2$ liberando \textbf{O$_2$} mentre il
gliossilato (è il più piccolo $\alpha$chetoacido) subisce la
transamminazione con il glutammato, producendo l'amminoacido
\textbf{glicina}.

\begin{itemize}
\itemsep1pt\parskip0pt\parsep0pt
\item
  \textbf{Reazione 2.6 e 2.7}
\end{itemize}

La glicina lascia il perossisoma ed entra nel mitocondrio dove un
complesso multienzimatico formato da \textbf{glicina decarbossilasi} e
\textbf{serina idrossimetiltransferasi} catalizza la reazione tra
\emph{due molecole di glicina} e \emph{una di NAD$^+$} per la produzione
di una molecola di \textbf{serina}, \textbf{NADH}, \textbf{NH$_4$$^+$} e
\textbf{CO$_2$}.

Ogni due glicolati si ottiene questo, con una conseguente importante
perdita netta di carbonio.

\begin{itemize}
\itemsep1pt\parskip0pt\parsep0pt
\item
  \textbf{Reazione 2.8}
\end{itemize}

La nuova serina formata diffonde dal mitocondrio verso i perossisomi
dove è convertita da una transamminazione a \textbf{idrossipiruvato}.

\begin{itemize}
\itemsep1pt\parskip0pt\parsep0pt
\item
  \textbf{Reazione 2.9}
\end{itemize}

L'idrossipiruvato è ridotto a \textbf{glicerato} con rilascio di NAD$^+$
tramite una \emph{reduttasi NADH-dipendente}.

\begin{itemize}
\itemsep1pt\parskip0pt\parsep0pt
\item
  \textbf{Reazione 2.10}
\end{itemize}

Il glicerato rientra nei cloroplasti dove viene fosforilato aformare il
\textbf{3-fosfoglicerato}.

In parallelo l'NH$_4$$^+$ rilasciato dall'ossidazione della glicina
diffonde dalla matrice del mitocondrio al cloroplasto, dove la
\textbf{glutammina sintetasi} catalizza la sua incorporazione
ATP-dipendente nel glutammato, formando \textbf{glutammina}.

\begin{itemize}
\itemsep1pt\parskip0pt\parsep0pt
\item
  \textbf{Reazione 2.12}
\end{itemize}

Successivamente una \textbf{glutammato sintetasi
ferredossina-dipendente} catalizza una reazione in cui reagiscono
\emph{glutammina} e \emph{2-oxoglutarato}, portando alla produzione di
due molecole di \textbf{glutammato}.

\textbf{IMMAGINE 8.8 p296}

L'equilibrio tra il ciclo di C-B e quello C$_2$ per l'ossidazione
fotosintetica del carbonio è determinato principalmente da tre fattori:
uno inerente alla pianta (le proprietà cinetiche della rubisco) e due
legati all'ambiente (la temperatura e la concentrazione dei substrati,
CO$_2$ e O$_2$).

Un aumento della temperatura esterna

\begin{itemize}
\itemsep1pt\parskip0pt\parsep0pt
\item
  modifica le costanti cinetiche della rubisco, eumentando la
  sercentuale di ossigenazione più di quella di carbossilazione;
\item
  diminuisce la concentrazione della CO$_2$ più di quella dell'O$_2$ in
  una soluzione in equilibrio con l'aria.
\end{itemize}

A temperature elevate dunque, l'aumentento della fororespirazione
rispetto alla fotosintesi limita notevolmente l'efficienza di
assimilazione fotosintetica del carbonio.

Se la rubisco funzionasse sempre, visto che il ciclo di Calvin consuma
ATP e NADPH che di notte non sono abbondanti, si dovrebbe trovare un
altro processo che produce queste molecole (la respirazione). Tuttavia,
se una pianta respira troppo consuma i prodotti della fotosintesi, di
notte quindi la rubisco deve essere ``spenta''.

Tutti gli enzimi hanno un'attività che risponde al pH (modificando il pH
modifichiamo anche l'enzima). Per qualsiasi enzima ci sono delle curve a
campana dalle quali si evince il valore del pH ottimale. Questi numeri
possono variare da enzima ad enzima.

Per la rubisco questo valore di pH è circa 8, più o meno il valore del
pH del citosol del citoplasto quando c'è trasporto di elettroni (in
condizioni di luce nelle quali la pianta sta producendo ATP e NADPH). La
rubisco funziona nel modo migliore ad un livello di pH che la pianta
raggiunge quando è esposta alla luce. Le rubisco hanno alcuni residui
amminoacidici perfettamente conservati, uno di questi è una lisina in
posizione 201; a pH basico quel gruppo amminico si comporta da base e
diventa un gruppo amminico libero. In questa forma una molecola di
CO$_2$ (non quella che viene fissata sul ribulosio) reagisce con il
gruppo $\epsilon$ della lisina andando a formare un carbammato (il
gruppo carbossilico è dissociato). La formazione del carbammato è
importante perchè la rubisco per legare i substrati ha bisogno di
incorporare un atomo di magnesio. Se tutto ciò non avviene l'enzima non
funziona. Se la rubisco non entra in questa serie di reazioni rimane
inattiva; non basta quindi il pH giusto.

Un altro meccanismo di regolazione è la rimozione della molecola di
substrato legata alla rubisco: la rubisco infatti non è attiva fin tanto
che non viene rimossa la molecola di substrato. La molecola di substrato
viene staccata dall'enzima grazie ad un enzima che prende il nome di
\textbf{rubisco attivasi}, richiede idrolisi di ATP.

La conversione della forma inattiva della rubisco in uno stato attivo,
necessaria sia per la reazione di carbossilazione che pper quella di
ossigenazione, richiede la carbammilazione di uno specifico residuo di
lisina. A seguito del legame della molecola di CO$_2$$^-$ con il residuo
di lisina, l'enzima acquisisce una triade di residui anionici che
forniscono il sito di legame per il cofattore Mg$^2$$^+$. A questo
stadio, l'incorporazione del ribulosio 1,5-BisP causa cambiamenti
conformazinali nella rubisco che permettono all'enzima di sequestrare lo
zucchero fosfato dal solvente e, a seguito del legame di una molecola di
CO$_2$ o di O$_2$, dare inizio al ciclo catalitico. Lo stretto legame
del ribulosio alla molecola non carbammilata dell'enzima sposta
l'equilibrio al vicolo cieco del complesso, rubisco-ribulosio, e fa sì
che la velocità di reazione diminuisca fino al completo arresto.

La \textbf{rubisco attivasi} disaccoppia il ribulosio dai siti attivi
decarbammilati e, così facendo, promuove l'accesso della CO$_2$ e del
Mg$^2$$^+$ per la carbammilazione dell'enzima. La veloctià alla quale la
rubisco attivasi aumenta la capacità catalitica della rubisco è legata
all'idrolisi di ATP (50 molecole di ATP idrolizzate per ogni sito di
rubisco attivato).

Nelle piante CAM i meccanismi di regolazione sono diversi. Nelle piante
C$_3$ la rubisco deve spegnersi di notte.

\subsubsection{Meccanismi di concentrazione del carbonio
inorganico}\label{meccanismi-di-concentrazione-del-carbonio-inorganico}

La Marcata riduzione di CO$_2$ e l'aumento dei livelli di O$_2$ che
iniziarono 350 milioni di anni fa hanno innescato una serie di
adattamenti per gestire un ambiente che promuove la fotorespirazione
negli organismi fotosintetici. Questi adattamenti comprendono varie
strategie per l'assorbimento attivo di CO$_2$ e HCO$_3$$^-$
dall'ambiente circostante e il conseguente accumulo di carbonio
inorganico da parte della rubisco.

La conseguenza immediata di questi meccanismi è la diminuzione della
reazione di ossigenazione.

Sono due i meccanismi sviluppati dalle piante terrestri per concentrare
CO$_2$ al sito di carbossilazione della rubisco:

\begin{itemize}
\itemsep1pt\parskip0pt\parsep0pt
\item
  la fissazione fotosintetica del carbonio \textbf{C$_4$};
\item
  il metabolismo acido delle Crassulacee \textbf{(CAM)}.
\end{itemize}

\paragraph{C4}\label{c4}

Le piante C4, di cui le più tipiche sono il mais, la canna da zucchero,
arundodonax, il miglio panicum, alcune chenopodiace, ecc\ldots{} sono
state per lo più introdotte nei nostri paesi, perché sono tutte
originarie dei tropici.

Questa via metabolica si svolge in due tipi morfologicamente distinti di
cellule - quelle del mesofillo e quelle della guaina del fascio - che
sono separati dalle loro rispettive membrane.

Nel cilo C$_4$, l'enzima \textbf{fosfoenolpiruvato carbossilasi
(PEPCasi)}, piuttosto che la rubisco, catalizza la carbossilazione
primaria in un tessuto che si trova vicino all'ambiente esterno. Il
conseguente flusso di acido a 4 atomi di C attraverso la barriera di
diffusione verso la regione vascolare, dove viene decarbossilato, libera
CO$_2$ che è ri-fissata dalla rubisco tramite il ciclo di C-B.

Le piante C4 fotorespirano poco. Questo tipo di piante aprono gli stomi
di giorno (come tutte le piante verdi), però non attuano il ciclo di
Calvin. Le cellule del mesofillo in queste piante contengono l'enzima
rubisco, il quele però è represso (non attivo).

Il trasporto della CO$_2$ dall'atmosfera esterna alle cellule della
guaina del fascio procede attraverso 5 fasi successive:

\begin{enumerate}
\def\labelenumi{\arabic{enumi}.}
\item
  Quando una molecola di CO$_2$ entra nella pianta, questa viene fissata
  nel mesofillo attraverso la carbossilazione del PEP (ad opera della
  PEPCasi) in \textbf{ossalacetato}. L'ossalacetato è successivamente
  convertito in \textbf{malato} dalla \textbf{malato deidrogenasi
  NADP-dipendente}, o convertito in \textbf{aspartato};
\item
  Successivamente si ha il trasporto di acidi 4C verso le cellule della
  guaina del fascio che circondano i fasci vascolari;
\item
  Si ha decarbossilazione degli acidi 4C e generazione della CO$_2$ che
  viene quindi ridotta a carboidrati attraverso il ciclo di C-B;
\item
  Poi si ha trasporto dello scheletro 3C (piruvato o alanina) formato
  dalla decarbossilazione verso le cellule del mesofillo;
\item
  In ultimo si ha la rigenerazione dell'accettore della CO$_2$: il
  piruvato viene convertito in fosfoenolpiruvato nella reazione
  catalizzata dalla \textbf{piruvato-fosfato dichinasi}. In questa
  reazione è richiesta una molecola supplementare di ATP per la
  trasformazione dell'AMP in ADP catalizzata dalla adenilaot chinasi.
\end{enumerate}

La compartimentazione degli enzimi assicura che il carbonio inorganico
dall'ambiente circostante possa essere assorbito inizialmente dalle
cellule del mesofillo, fissato successivamente dal ciclo di C-B nelle
cellule della guaina del fascio e infine esportato verso il floema.

Nelle piante C$_3$ la guaina del fascio è formata da 1-2 strati di
cellule, mentre nelle piante C$_4$ è molto più sviluppata. Separando le
cellule della guaina C$_4$ si scopre una forte attività della rubisco,
che in queste piante si trova compartimentalizzata solo in questo tipo
di cellule.

Nelle piante troviamo i \textbf{plasmodesmi}; queste sono strutture
attraverso le quali vengono scambiate molecole. Queste molecole si
spostano tramite diffusione da zone in cui si trovano a concentrazioni
maggiori a zone a concentrazioni minori. Per questo motivo le molecole
formate nel mesofillo si spostano nella guaina del fascio dove si ha
decarbossilazione e infine azione dell'enzima rubisco. Questo meccanismo
blocca la reazione di ossigenazione della rubisco.

Se io misuro la concentraizone della CO$_2$ nella guaina del fascio
scopro che è molto più alta di quanto ci si aspetterebbe. Quando il
rapporto CO$_2$/O$_2$ si alza la velocità di ossigenazione si abbassa.

Questa alta concentrazione di CO$_2$ inibisce l'enzima rubisco (metodo
istologico di inattivazione dell'enzima rubisco evoluto dopo la presenza
di O$_2$ nell'atmosfera).

\textbf{(Perchè inibisce l'enzima?)}

Perché non tutte le piante sono C4 se funziona così bene? Perché
traslocare il malato costa molte moleocle di ATP. L'ATP è un fattore
limitante per una pianta dove c'è poca luce. Per questo non tutte le
piante possono sfruttare questo meccanismo (complessivamente costa molto
di più del meccanismo di Calvin). Queste piante posono crescere solo in
ambienti in cui la luce non è un fattore limitante.

\paragraph{CAM}\label{cam}

Le piante solitamente tengono aperti gli stomi di giorno e li chiudono
di notte mentre le CAM fanno l'opposto, lasciando entrare la CO$_2$ di
notte quando la fotosintesi non avviene.

Da un punto di vista organolettico le CAM possono avere sapori diversi a
seconda che le piante provengano da una fase lumiosa o di buio; se
vengono dalla fase di buio hanno un sapore acido, mentre se vengono
dalla fase luminosa hanno un sapore zuccherino.

Le piante CAM accumulano acqua nei vacuoli (sono molto grandi, il
90-95\% del volume cellulare è occupato da questi).

Di notte quello che succede è che il fosfoenol piruvato, tramite
l'enzima \emph{fosfoenolpiruvato carbossilasi}, media una
carbossilazione formando ossalacetato. Il primo prodotto che compare
nella fotosintesi è una molecola a 4C. Eventualmente un gruppo
carbonilico può venire ridotto dando l'\textbf{acido malico} (malato,
anione). A questo punto le CAM traslocano tutto il malato nei vaucoli
(siamo al buio) riempiendoli, qui il malato raggiunge concentrazioni
millimolari (è tanto). Per questo di mattina i cactus, se li assaggiamo,
sono acidi.

Di giorno invece le CAM chiudono gli stomi e la CO$_2$ non entra più
nella pianta (questa strategia è attuata perché di giorno fa caldo e
l'apertura degli stomi favorisce la traspiraizone, che in queste piante
che vivono nel deserto sarebbe eccessiva e dunque dannosa). Di giorno
l'acido malico esce dai vacuoli e alcuni enzimi lo decarbossilano
(enzima \textbf{malico} e \textbf{PEP decarbossichinasi}). Di giorno i
vacuoli dunque liberano il malato che viene decarbossilato a
\emph{piruvato} e CO$_2$ che entra nel ciclo di Calvin (non può uscire
dalla pianta perché gli stomi sono chiusi). Di giorno dunque la pianta
produce zuccheri e per questo diventa dolce.

\textbf{IMMAGINE 8.13 p311} CARBONIO

Queste piante crescono lentamente perchè la quantità di carbonio che
possono accumulare ogni giorno e utilizzare per la crescita non è molto
elevata.

\subsection{Metabolismo del carbonio nelle cellule degli
stomi}\label{metabolismo-del-carbonio-nelle-cellule-degli-stomi}

Gli stomi sono una parte molto importante delle piante perché sono
l'unico punto in cui l'ambiente interno delle piante entra in contatto
con l'ambiente esterno. Gli stomi regolano gli scambi tra la pianta e
l'ambiente.

Gli stomi sono formati da due cellule molto specializzate la cui
peculiarità deriva soprattutto dalla loro parete cellulare. Il ruolo
fondamentale degli stomi è quella di far entrare la CO$_2$. Senza stomi
le piante sarebbe quasi totalmente impermeabili e dunque gli scambi
gassosi sarebbero troppo limitati. Non appena gli stomi si aprono per
far entrare la CO$_2$ però si ha una contemporanea perdita di H$_2$O
(può essere molto elevata e dannosa per la pianta).

Se io lascio una pentola aperta con dentro dell'acqua questa evaporerà a
causa dell'umidità inferiore dell'ambiente che la circonda.

Tutte le ragioni che portano ad un movimento dell'acqua vengono
riassunte nella funzione del potenziale idrico:

\ldots{} = f (C, P, $\pi$, h)

Dove:

\begin{itemize}
\itemsep1pt\parskip0pt\parsep0pt
\item
  \textbf{C} è la concentrazione;
\item
  \textbf{P} è la pressione;
\item
  \textbf{$\pi$} è la componente osmotica;
\item
  \textbf{h} è l'altitudine.
\end{itemize}

La pressione atmosferica è sufficiente per far salire l'acqua in un
capillare di circa 10 metri. Perciò la pressione di cui si parla per far
salire l'acqua lungo i vasi di una pianta molto alta non è la pressione
atmosferica. La pianta deve attuare un compromesso tra respirazione e
fotosintesi; per ogni pianta posso definire quante moli di H$_2$O
vengono perse per ogni mole di CO$_2$ che vengono prese. E' l'unico
punto di controllo della quantità di H$_2$O della pianta.

Gli stomi hanno una parete inspessita in maniera diversa tra la zona
centrale e quella esterna dello stoma. C'è una forza che tende a
deformare la cellula ma in modo diverso perchè la parte interna è più
inspessita rispetto a quella esterna. Così riesce a inglobare acqua
quando è sotto pressione.

Nelle piante oltre alla clorofilla ci sono altrre molecole che assorbono
la luce: queste molecole non hanno una funzione fotosintetica ma
segnale. Il movimento delle piante è una variazione della struttura
nello spazio e nel tempo. Questi movimenti sono mediati dalla luce che
viene percpepita dalle \textbf{fototropine} che inducono ``movimenti''.

La luce negli stomi è percepita dalle fototropine; queste, direttamente
o indirettamente, hanno come bersaglio molecolare della pompe capaci di
traslocare ioni da una parte all'altra della membrana utilizzando ATP.
Le piante usano principalmente ioni H$^+$ all'esterno della cellula.
Oltre a un $\delta$pH si forma anche una differenza di potenziale che
apre i canali del K$^+$ che a questo punto entra nella cellula.

Contemporaneamente i cloroplasti di queste \emph{cellule guardia}
possono fare varie cose: l'amido nei cloroplasti comincia ad essere
degradato, e i triosi fosfati possono uscire dal cloroplasto attraverso
un meccanismo C$_4$ (si forma malato). Entrambi gli ioni, quelli
positivi del K$^+$ e quelli negativi del malato, vengono traslocati nel
vacuolo. Il risultato della percezione della luce viene trasformato
dentro gli stomi e porta all'accumulo di malato e K$^+$ .
Contemporaneamente, l'aumenta della concentrazione dei soluti nel
vacuolo porta al richiamo di acqua dal citoplasma.

Questo porta ad un aumento della pressione dentro le cellule e ciò fa sì
che lo stoma si apra. La forza necessaria a deformare la cellula è
dovuta all'accumulo di malato nei vacuoli. Affinchè lo stoma stoma si
chiuda invece, servono delle ATPasi che devono smettere di funzionare.

Ci sono due modi per chiudere gli stomi:

\begin{enumerate}
\def\labelenumi{\arabic{enumi}.}
\itemsep1pt\parskip0pt\parsep0pt
\item
  tramite inattivazione delle fototropine, come avviene di notte;
\item
  tramite controllo da parte dell'\textbf{acido abscissico} (un ormone)
  che blocca o comunque fa variare il funzionamento delle ATPasi. Questo
  meccanismo funziona anche di giorno.
\end{enumerate}

C'è un altro parametro, ossia la concentrazione di CO$_2$ dentro la
pianta. L'apertura degli stomi è infatti stimolata da una bassa
concentrazione di CO$_2$ mentre la chiusura dipende da un'alta
concentrazione di CO$_2$.

\textbf{LEZIONE 03.11.2015}

\subsection{L'efficienza fotosintetica delle
piante}\label{lefficienza-fotosintetica-delle-piante}

Le piante C$_3$ e C$_4$ in sezione sono molto diverse; la differenza
principale sta nelle cellule disposte intorno ai fasci cribrovascolari.
Nelle piante C$_3$ si trova un singolo strato di cellule, mentre nelle
piante C$_4$ ci sono altri strati di cellule.

Le strutture si chiamano \textbf{guaina del fascio} e metabolizzano la
C4. Era considerato anche come carattere tassonomico. L'anatomia è stata
scoperta molto prima della fotosisntesi C4.

Separando le cellule del mesofillo da quelle della guaina del fascio per
mezzo di enzimi che degradano le pareti cellulari, e studiando gli
enzimi carbossilattivi, si scoprì che nelle guaine del fascio l'enzima
rubisco era ristretto alle cellule della guaina del fascio delle piante
C$_4$. La CO$_2$ una volta entrata in una pianta C$_4$ non incontra
immediatamente la rubisco. Quando una molecola di CO$_2$ incontra la
PEPcarbossilasi, un enzima che dà varie vie fotosintetiche e catalizza
una reazione in cui i PEP funge da accettore per una molecola di CO$_2$,
diventa ossalacetato. L'ossalacetato entra nei cloroplasti e il suo
gruppo carbonilico viene ridotto trasformando la molecola in malato. Il
malato viene spostato nelle cellule della guaina del fascio tramite i
plasmodesmi. Nelle cellule della guaina si trova la rubisco; il malato
viene decarbossilato e libera la CO$_2$ che entra nel ciclo di Calvin.
Il ciclo C$_4$ sposta la CO$_2$ dal mesofillo alle guaine del fascio
sotto forma di malato, e il malato in eccesso torna sempre via
plasmodesmi al mesofillo dove viene riconvertito a PEP. Per traslocare
una molecola di CO$_2$ dal mesofillo alle guaine del fascio le piante
spendono due molecole di ATP ed una di pirofosfato (PP$_i$) più le moli
di NADPH e ADP. In totale le piante C$_4$ spendono di più delle C$_3$
per compiere la fotosintesi.

La CO$_2$ entra nelle piante in funzione delle concentrazioni. Se misuro
la CO$_2$ in un meccanismo C$_4$ noto che la sua concentrazione è molto
alta. Il meccanismo di trasporto del malato serve per concentrare la
CO$_2$ nei tessuti dove c'è la rubisco. Il meccanismo porta un aumento
del rapporto tra
CO\$\emph{$2 e ossigeno inibendo l’ossigenazione. La fotosintesi C$}4\$
diventa conveniente quando la concentrazione di ATP è abbondante, ossia
in presenza di luce. Queste sono le condizioni che troviamo ai tropici.

Il malato viene trasportato e si forma piruvato che poi torna nel
mesofillo per essere riconvertito a fosfoenolpiruvato. Questa non è una
reazione favorita. Le piante hanno un secondo enzima per svolgere questa
funzione, la \textbf{piruvato-fosfato-dichinasi (PPDK)}. Le dichinasi
compiono contemporaneamente due fosforilazioni ovvero sfruttano
contemporaneamente 2 gruppi fosfato di 2 molecole di ATP: il primo
gruppo fosfato serve a fosforilare il fosfato \textbf{(??)}, mentre il
secondo gruppo serve per fosforilare il piruvato a fosfoenolpiruvato. Il
pirofosfato, una volta formatosi, viene immediatamente eliminato da un
enzima che lo idrolizza in due fosfati. È l'enzima
\textbf{pirofosfatasi}. Ogni volta che dall'equilibrio tolgo un po' di
prodotto spingo la reazione nella direzione dei prodotti; questo
significa che eliminando il pirofosfato automaticamente verrà prodotto
più PEP.

Il fatto che le piante C$_4$ siano molto diverse tra loro ci fa pensare
che questo sia un gruppo polifiletico. Questi gruppi di piante si
differenziano nel meccanismo in cui operano la decarbossilazione
dell'atomo della molecola C$_4$. Possono dercabossilare il malato in due
modi diversi:

\begin{itemize}
\item
  tramite una \emph{decarbossilasi del malato} che produce CO$_2$ e
  NADPH con enzima malico, come nel mais. Il gruppo alcolico è ossidato
  a carbossile e si liberano elettroni che vengono raccolti dal NADP$^+$
  che diventa NADPH. Il malato trasporta CO$_2$ ma anche equivalenti
  riducenti;
\item
  tramite \textbf{fosfoenolpiruvato carbossichinasi}, un enzima che
  decarbossila e sintetizza ATP. Il sorgo ad esempio è una tipica pianta
  C$_4$ che usa questo enzima. Le piante come il miglio anziché usare il
  malato come molecola di trasporto usano l'aspartato.
\end{itemize}

Tutte queste piante hanno origini diverse. La fotosintesi C$_4$ ha la
funzione di aprire gli stomi e di impedire la fotorespirazione, mentre
nelle piante CAM ha lo scopo di adattare la pianta alla vita.

\subsubsection{Formazione e accumulo dell'amido e del
saccarosio}\label{formazione-e-accumulo-dellamido-e-del-saccarosio}

In una sezione di foglia di mais si vedono granuli di amido in
prossimità dei cloroplasti perché la fotosintesi avviene nelle guaine
del fascio.

L'amido non è un omopolimero del glucosio. Esso è formato da
\textbf{amilosio}, \textbf{amilopectina} e anche \textbf{ficoglicogeno}.
Presemta legami $\alpha$ 1 $\rightarrow$ 4 e ramificazioni $\alpha$ 1
$\rightarrow$ 6.

I granuli di amido crescono tramite depozione di strati concentrici
attorno ad un nucleo centrale

L'amido può essere sintetizzato a partire dal \textbf{fruttosio
6-fosfato} (un intermedio del ciclo di Calvin), per mezzo
dell'\textbf{esoso fosfato isomerasi} che trasforma una parte del
fruttosio in \textbf{glucosio 6-fosfato}. Con un'altra reazione il
glucosio 6P può essere mutato in \textbf{glucosio 1-fosfato}. Queste
molecole costituiscono il pool degli esoso fosfati.

Questo può avvenire a concentrazioni elevate di fruttosio 6P. Il
glucosio 1P può reagire con l'ATP per mezzo di una reazione critica in
cui un'adenina e il terzo gruppo fosfato di una molecola di ATP sono
condensati con il glucosio 1P producendo \textbf{adenosina
difosfoglucosio (ADPG)} e \textbf{pirofosfato}. Il pirofosfato viene
idrolizzato via \textbf{pirofosfatasi} e spinge l'equilibrio verso la
produzione di ATP.

Il glucosio in questa forma è molto ricco di energia, se nel cloroplasto
vie è una molecola di amido con \emph{n} residui può reagire con questa
molecola e dalla reazione si libera ADP. Questo glucosio reagisce con
l'amido solo perché una volta fosforilato è una molecola molto attiva.
L'enzima chiave è l'\textbf{amido sintetasi}.

Se il rapporto fosfoglicerato/fosfato è alto l'\textbf{ADPG fosforilasi}
è attiva (cioè quando la pianta è a corto di fosfato).

L'envelope dei cloroplasti è una membrana molto complicata derivante da
un processo di endosimbiosi. Tra i vari meccanismi con cui i cloroplasti
e il resto della cellula comunicano esistono tantissimi traslocatori.
Uno tra tutti è il \textbf{traslocatore dei triosifosfato}, un
antiporto. Questo trasportatore esporta triosi fosfati in cambio di
fosfato. Il ciclo di Calvin consuma anche fosfato perché il suo prodotto
finale è un trioso fosfato (il fosfato che entra alimenta il ciclo di
Calvin). Quando la pianta è a corto di fosfato produce amido per rendere
disponibile il fosfato che ha all'interno. Il traslocatore funziona come
bilanciatore. Se non c'è fosfato le piante producono amido così da
rilasciare fosfato, se invece il fosfato è presente esportano triosi. La
foglia al buio è piena di amido, alla luce no. Quando la pianta compie
tanta fotosintesi accumula amido per avere fosfati, quando la
fotosintesi cessa l'amido viene degradato. Una pianta in condizioni
estremamente favorevoli non accumula amido. Gli amiloplasti sono le
riserve di amido che non è accumulato nelle foglie ma per esempio nelle
radici. Le foglie hanno solo il compito di produrlo.

Quando i triosi fosfato escono dal cloroplasto vanno a formare
saccarosio. Il saccarosio è trasportato dal floema. Non tutte le piante
sintetizzano saccarosio, molte piante trasportano zuccheri più
complicati.

I triosi fosfati vanno nel citoplasma delle cellule che è dotato di vari
glucosi e fruttosi (la loro differenza dipende dalla loro origine e
dalla reazione tramite cui sono stati formati). Il glucosio qui reagisce
con \textbf{uridina trifosfato} e si ha \textbf{uridina difosfoglucosio}
e \textbf{difosfato} che però non viene idrolizzato perché nel citosol
non c'è lo stesso enzima che c'è nel cloroplasto. La molecola di
glucosio attivata raggiunge come recettore una molecola di fruttosio 6
fosfato e si forma \textbf{saccarosio 6-fosfato}. Una fosfatasi
defosforila la molecola e rende irreversibile il processo. È questo
enzima che ha lo stesso ruolo della pirofosfatasi nella sintesi
dell'amido.

Il fosfato viene riciclato più volte nel ciclo di Calvin; ha una
funzione catalitica ma se presente in abbondanza vi è un'export di atomi
di carbonio verso il citosol. I prodotti della fotosintesi ossia i
fotosintati come il saccarosio vengono trasportati in vari distretti
della pianta. Le foglie sono una sorta di sorgente di zuccheri, mentre i
frutti sono una sorta di pozzo metabolico degli zuccheri. Le piante
accumulano anche amido negli amiloplasti. Il ruolo del tessuto
floematico è quello di collegamento. Nel floema la linfa va in tutte le
direzioni disponibili, non vi è una corrente direzionale unica.

Il trasporto è regolato dai principi della fisica. Si parla di
diffusione: la diffusione fa sì che un soluto sia presente in eguale
concentrazione in tutte le parti. Questo tipo di movimento avviene in
risposta ad un gradiente di concentrazione. Questo processo è descritto
dalla \textbf{legge di Fick} che nella sua forma più semplice stabilisce
una relazione secondo cui un flusso (quantità di materia che si muove su
una superficie) di una sostanza dipende dal coefficiente di flusso che
dipende dalla molecola e dalla differenza di concentrazione.
\textbf{J$_s$ = -D dC/dx}

Questa legge lega la velocità con cui le molecole si muovono in funzione
del suo gradiente di concentrazione. L'equazione integrata arriva a
\textbf{t = (distanza$^2$/D) K}

C'è una relazione quadratica tra la distanza percorsa nel tempo. Una
molecola d'acqua impiega circa 2,5s a spostarsi da una parte all'altra
della cellula, e impiega circa 24 anni per coprire una distanza di un
metro.

\subsection{Assimilazione deinutrienti
minerali}\label{assimilazione-deinutrienti-minerali}

Le piante, oltre che di C, O e H necessitano anche di nutrienti minerali
quali azoto (N) e zolfo (S).

Quando si parla di \emph{``macronutrienti''} si intendono tutti quei
nutrienti necessari alla pianta in grande quantità (es. S, Ca e K),
mentre con \emph{``micronutrienti''} si intendono quei nutrienti
presenti in tracce (es. metalli di transizione che hanno stati di
ossidazione diversi).

Per scoprire quali sono gli elementi indispensabili per la crescita
della pianta si fanno germinare i semi in una ``soluzione'' (anzichè
nella terra) dove si può contrallare con precisione quali elementi sono
presenti e in quali quantità. Se l'elemento che decidiamo di non
somministrare non è indispensabile la pianta riuscirà ugualmente a
crescere, mentre se l'elemento è indispensabile per la vita della pianta
senza di questo essa non riuscirà a crescere.

\subsubsection{Il metabolismo
dell'azoto}\label{il-metabolismo-dellazoto}

L'azoto, dopo il C e l'O, è l'elemento più importante poichè
indispensabile per molte molecole indispensabili come basi azotate,
amminoacidi, pigmenti\ldots{}

Negli amminoacidi l'azoto si trova sotto forma amminica, mentre nella
clorofilla fa parte dell'anello pirrolico (è sempre un gruppo amminico).
L'azoto è sempre presente sotto forma amminica.

L'azoto con stato di ossidazione 0 ha 5 elettroni nello strato più
esterno. Nel nitrato (n.o. +5) i 5 elettroni sono stati persi, mentre
nell'ammonio (n.o. -3) l'azoto ha acquistati 3 eletroni. Sia il nitrato
che l'ammonio completano l'ottetto (diventando più stabili). Ci sono 8
elettroni di differenza tra le due molecole.

L'azoto amminico è stato selezionato dall'evoluzione.

L'atmosfera contiene grandi quantità di azoto molecolare (N$_2$).
L'acquisizione di azoto dall'atmosfera richiede la rottura di un triplo
legame covalente, eccezionalmente stabile, presete tra due atomi di
azoto per produrre ammoniaca(NH$_3$) o nitrato (NO$_3$$^-$). Una volta
che l'azoto è stato fissato in ammoniaca o nitrato, entra nel ciclo
biogeochimico e passa attraverso diverse forme organiche ed inorganiche
prima di ritornare alla forma molecolare. In condizioni di elevata
concentrazione nel suolo, come accade dopo la fertilizzazione,
làassorbimento dell'ammonio e del nitrato da parte delle radici può
eccedere la capacità della pianta di assimilazione, portando ad un
accumulo degli ioni nei tessuti vegetali. Poichè alte concentrazioni di
ammonio nei tessuti risultano tossiche per le piante (e anche per gli
animali), gli eccessi di questa molecola dopo essere stati assimilati
vengono velocemente accumulati nei vacuoli. Elevate concentrazioni di
nitrato invece non hanno effetti deleteri sulla pianta e possono essere
traslocate da tessuto a tessuto.

In generale l'azoto nel terreno si trova sotto forma di nitrato, poichè
vi sono batteri capaci di ossidare l'azoto organico in nitrato. Le
piante quindi hanno a disposizione nitrati, a meno che non si vada nelle
paludi dove si generano condizioni anaerobiche e non si ha l'ossigeno
come accettore.

Trasformare il nitrato in ammonio significa investire 8 elettroni (2 per
il passaggio della CO$_2$ a carboidrato). Assimilare l'azoto è dunque un
processo energeticamente molto costoso.

La prima cosa che le piante devono fare è assorbire l'N. Il nitrato
entra attraverso una

Le radici delle piante assorbono attivamente il nitrato dalla soluzione
del suolo tramire numerosi cotrasportatori nitrato-protone; questi
enzimi sono chiamati \emph{permeasi} e formano un canale che permette
l'entrata di certe molecole. Le permeasi pompano protoni all'esterno
che, rientrando, si portano dietro un nitrato (è un processo attivo che
dipende dal pH).

Il primo passafffio di questi processo è la riduzione citosolica del
nitrato in nitrito in una reazione che comporta il trasferimento di
\emph{due elettroni}. L'enzima \textbf{nitrato reduttasi} catalizza una
reazione di questo tipo:

\textbf{NO$_3$$^-$ + NAD(P)H + H$^+$ $\longleftrightarrow$ NO$_2$$^-$ +
NAD(P)$^+$ + H$_2$O}

La forma più comune di nitrato reduttasi utilizza solo NADH come
donatore di elettroni, mentre un'altra forma dell'enzima che si trova
principalmente in tessuti non verdi (come le radici), può utilizzare sia
NADH che NADPH.

Un eccessivo accumulo di nitrito è tossico per le cellule e per questo
deve essere immediatamente smaltito. Il nitrito dunque, viene
immediatamente traslocato nei plastidi delle cellule della radice, detti
\textbf{leucoplasti}.

Nelle cellule meristematiche i plastidi vegono detti
\textbf{proplastidi}. Questi possono specializzarsi in maniera diversa e
diventare cloroplasti (nelle foglie), amiloplasti (se conservano amido),
cromoplasti (nei frutti)\ldots{} Possono trasformarsi abbastanza
agevoltente anche dopo essersi specializzati già una prima volta (hanno
capacità plastica).

Una volta nei leucoplasti la nitrito reduttasi riduce il nitrito in ione
ammonio utilizzando 3 NADPH. Questo NADPH proviene dal ciclo dei pentosi
fosfati.

\textbf{NO$_2$$^-$ + 6 Fd$_rid$ + 8 H$^+$ $\longleftrightarrow$
NH$_4$$^+$ + 6 Fd$_oss$ + 2 H$_2$O}

La ferredossina ridotta deriva dal trasporto elettronico forosintetico
nei cloroplasti e dal NADPH generato dalla via ossidativa dei pentosi
fosfati nei tessuti non verdi. I cloroplasti e i plastidi della radice
contengono diverse forme dell'enzima, ma entrambe le forme consistono di
un singolo polipeptide che contiene due gruppi prostetici: un centro
ferro-zoldo (Fe$_4$S$_4$) ed un eme specializzato. Questi gruppi
agiscono insieme per legare il nitrito e lo riducono direttamente ad
ammonio. Una piccola percentuale di nitrato ridotto è liberata come
ossido nitroso (N$_2$O).

Alla fine questo azoto viene ritrovato sotto forma di ammidi (un gruppo
carbossilico che reagisce con un gruppo ammoniaca). Piante diversa
produco ammidi diversi. Questi ammidi escono dal leucoplasto e tramite
lo xilema vengono trasportati alle parti aeree delle piante.

Alternativamente, per esempio nelle piante a basso fusto, il nitrato
passa dalle foglie per mezzo dello xilema e nelle cellule del mesofillo
ha un destito simile a quello che avrebbe nelle radici. Il nitrito entra
pero' nei cloroplasti dove il potere riducente grazie alla fotosintesi
e' alto e viene ridotto da una vairante della nitrato riduttasi che
sfriutta la ferrodoxina. Si producono amminoacidi. Qui l'assimilazione
del nitrato e' fotodipendente perche' dipende dai reagenti della fase
luce della fotosintesi. I vacuoli possono funzionare da organi
temporanei di accumolo dei nitrati.

\paragraph{La nitrato reduttasi}\label{la-nitrato-reduttasi}

Le nitrato reduttasi delle piante superiori sono \emph{omodimeri},
composte cioè da due subunità identiche, ognuna con tre domini
funzionali ciascuno dei quali contiene un centro redox diverso:

\begin{enumerate}
\def\labelenumi{\arabic{enumi}.}
\itemsep1pt\parskip0pt\parsep0pt
\item
  un \textbf{FAD};
\item
  un \textbf{eme};
\item
  un \textbf{molibdeno (MoCO, molibdeno cofactor)} complessato con una
  molecola organica detta \emph{pterina}.
\end{enumerate}

Il dominio che lega il FAD accetta due elettroni dal NADH. Gli elettroni
passano quandi attraverso il dominio dellàeme per giungere al complesso
molibdeno, dove sono trasferiti al nitrato che diventa nitrito.

Il molibdeno può distorcere la struttura dell'atomo variandone il
potenziale redox e così facendo rende accessibile ai riducenti biologici
quello che non lo è. Per questo motivo l'N è spesso legato ad esso.

Il NAD non riduce direttamente il nitrato ma trasferisce elettroni
all'accetore FAD, il quale li trasferisce ad un \textbf{citrocromo
b557}. Questo citocromo riduce il cofattore molibdeno che a sua volta
riduce il nitrato a nitrito essendo in grado di intergire con lo stesso.
Il complesso ha dunque da una parte il sito di legame per il NAD, e
dell'altra quello per il nitrato (agisce come una catena elettronica).

La nitrato reduttasi è un enzima complicato fatto da due domini
funzionali, uno dei quali lega la ferredossinache. LA ferredossina
trasferisce elettroni da un citocromo, il quale a sua volta riduce il
nitrito ad ammonio. Il processo avviene in due step e costa una molecola
di NADH/3 e NADH/6 ferrodoxina.

L'ammonio non rappresenta il prodotto finale della reazione, ma è un
disaccoppiante che inibisce la sintesi di ATP nei citocromi agendo sulla
catena elettronica. Le piante hanno meccanismi raffinati per recuperare
l'ammonio della via fotorespiratoria o riciclare quello delle radici.

\paragraph{Assimilazione dell'ammonio}\label{assimilazione-dellammonio}

Le cellule vegetali evitano la tossicità dell'ammonio convertendo
rapidamente l'ammonio generato dall'assimilazione del nitrato o dalla
fotorespirazione in amminoacidi.

La \textbf{glutammina sintetasi (GS)} unisce l'ammonio con il glutammato
per formare la \textbf{glutammina}.

\textbf{Glutammato + NH$_4$$^+$ + ATP $\longleftrightarrow$ glutammina +
ADP + P$_i$}

Elevate quantità di glutammina stimolano l'attività della
\textbf{glutammato sintasi}

La glutammato sintasi, in presenza di alfa-chetoglutarato
(alfa-chetoacido a 5 atomi di C) che funziona da accettore per
l'ammonio, produce glutammato (trasferisce un gruppo amminico). Questa
reazione richiede la presenza di ferredossina (agente riduncente) e di
ATP.

Questo enzima trasferisce il gruppo ammidico della glutammina al
2-oxoglutarato, formando due molecole di glutammato.

\textbf{Glutammina + 2-oxoglutarato + NADH + H$^+$ \$ + Fd$_rid$
$\longleftrightarrow$ glutammato + NAD$^+$ + Fd$_oss$}

Complessivamente questo viene definito \textbf{``sistema GS/GOGAT''}.

\paragraph{La fissazione dell'azoto}\label{la-fissazione-dellazoto}

A partire da azoto atmosferico i chimici compiono una reazione di
fissazione in cui azoto e idrogeno producono ammoniaca con liberazione
di energia (quindi reazione favorita). In condizioni biologiche
l'energia di attivazione è molto alta, quindi questa reazione non
avviene. Questa è una fissazione a-biologica dell'azoto e può essere
confrontata con quella biologica che è quella compiuta dai batteri
azoto-fissatori.

L'enzima che catalizza la fissazione dell'azoto è la
\textbf{nitrogenasi}. Per questa reazione sono necessarie 12 molecole di
ATP, 6 elettroni e 8 H+, mentre i prodotti sono 2 molecole di
NH$_4$$^+$, 12 ADP e 12 P$_i$.

Contemporaneamente l'enzima catalizza una reazione di questo tipo: 4 ATP
+ 2 e$^-$ + 2 H$^+$ $\longleftrightarrow$ H$_2$ + 4 ADP + 4 P$_i$ Questo
rende la reazione estremamente costosa, poichè in totale sono necessarie
16 molecole di ATP. Questa è l'unica reazione che esiste sul nostro
pianeta capace di trasformare N$_2$ in NH$_4$$^+$.

Gli enzimi della nitrogenasi che catalizzano queste reazioni possiedono
siti che facilitano lo scambio di elettroni ad alta energia. L'ossigeno,
essendo un forte accettore di elettroni, può danneggiare questi siti e
disattivare irreversibilmente la nitorgenasi, così l'azoto deve essere
fissato in condizioni anaerobiche. Questo è il motivo per cui gli
organismi eucarioti non sono azotofissatori.

La fissazione dell'azoto è un processo biochimico che appartiene al
mondo dei procarioti. I procarioti che fissano azoto possono essere
liberi (come i cianobatteri) o possono vivere in simbiosi con piante
superiori (come i Rizhobium).

I cianobatteri, come Oscillatoria, sono batteri coloniali che formano
delle catenelle. Le cellule che formano queste catenelle sono dette
\emph{cianocisti}. Sparse tra le altre cellule si trovano anche cellule
più grosse dette \emph{eterocisti}. Queste cellule più grandi sono
dotate di una parete più spessa impermeabile ai gas, e non possiedono il
PSII (questo è il fotosistema che sviluppa ossigeno nei cloroplasti,
così non producono ossigeno). Sono invece cellule specializzate
nell'azoto fissazione e ricevo l'energia necessaria dalle cellule
adiacenti.

Altri batteri azotofissatori vivono in simbiosi con diverse specie
vegetali. Questi batteri formano dei noduli radicali e possiedono la
\textbf{leg-emoglobila} (si trova nei noduli), una proteina simile
all'emoglobina umana che lega l'ossigeno affinchè non questo non possa
inattivare la nitrogenasi. La leg-emglobina ha un'affinità estremamente
elevata per l'O$_2$ (lo lega anche quando ce n'è molto poco e lo
rilascia difficilmente).

In terreno oligotrofo, piante inoculate con i rizobi crescomo molto
meglio rispetto a piante non inoculate. Alcune simbiosi possono essere
quasi specie-specifiche. Questo presuppone un meccanismo di
riconoscimento molecolare tra le radici della piante e i batteri liberi
nel terreno.

Le radici delle piante trasudano molecole; tra queste vi sono delle
molecole chiamate \textbf{elicitori} che interagiscono con organismi
capaci di reagiranno alla loro presenza. Gli elicitori sono molecole
complesse appartenenti al gruppo dei flavonoidi. Dopo che il batterio ha
percepito la molecola rilasciata dalla pianta produce a sua volta dei
\textbf{fattori nod} che devono essere riconosciuti dalla pianta. La
percezione degli elicitori da parte dei batteri è un processo abbastanza
generico, mentre la percezione dei fattori nod da parte della pianta è
un meccanismo molto specifico.

Le leguminose, ad esempio, secernono \textbf{luteonina}. Se nel terreno
si trovano dei batteri questi possono percepire la luteonina, la quale
indurrà una risposta che attiverà l'espressione di alcuni geni dentro il
batterio. L'espressione di alcuni di questi geni determina la sintesi di
una molecola (un \emph{polimero di acetilglucosammina}) con una coda
lipidica che potrà essere nel suo insieme modificata e che viene
chiamata \textbf{lipopoligosaccaride}. Quello che varia del polimero è
la sostituzione dei residui di acetilglucosammina. Se la pianta
riconosce il batterio attraverso la struttura del lipopoligosaccaride
inizia la simbiosi, altrimenti no. Quando la pianta riconosce il
batterio in breve tempo si formano dei noduli contenenti la
leg-emoglolina. La leg-emoglobina è formata dalla pianta (la pianta
fornisce anche gli scheletri carboniosi di tutto ciò che viene formato),
mentre il batterio sintetizza la nitrogenasi.

\subsubsection{Il metabolismo dello
zolfo}\label{il-metabolismo-dello-zolfo}

Lo zolfo è un elemento riscontrabile in alcune molecole molto importanti
come in alcuni amminoacidi, nel CoA, in alcune vitamine\ldots{}

La maggior parte dello zolfo presente nei costituenti carboniosi delle
cellule delle piante superiori deriva principalmente dal solfato
(SO$_4$$^2$$^-$), assorbito tramite un simportatore H$^+$-SO$_4$$^2$$^-$
dalla soluzione del suolo.

Da un punto di vista redox lo zolfo è un ``cattivo cliente''. Tra il
gruppo solfato e quello sulfidrile (presente nelle biomolecole) ci sono
8 elettroni di differenza. Assimilare lo zolfo è dunque un processo
molto costoso perche nel passaggio da solfato a sulfidrile è necessario
aggiungere questi 8 elettroni. La coppia redox ha un potenziale standard
di ossidoriduzione di -400 mV. La coppia solfato/solfito ha un
potenziale standard di -517 mV. Nessun reagente può ridurre direttamente
il solfato.

La prima tappa nel processo di assimilazione dello zolfo è l'attivazione
del gruppo solfato il quale è molto stabile. L'attivazione inizia con la
reazione fra il solfato e l'ATP per formare una molecola di
\textbf{adenosina-5'-fosfosolfato (APS)} (il solfato va a sostituire i 2
gruppi fosfati, $\gamma$ e $\beta$, dell'ATP) e una di
\textbf{pirofosfato (PP$_i$)}. L'enzima che catalizza questa reazione è
chiamato \textbf{ATP-sulfurilasi}.

\textbf{lezione 09/11}

\subsubsection{L'acqua nelle piante (AGGIUNGERE? confrontare con le
altre)}\label{lacqua-nelle-piante-aggiungere-confrontare-con-le-altre}

La legge di Fick spiega come si determinano i flussi dei soluti e la
diffusione è un processo passivo regolato da questa legge. Questa legge
riguarda una determinata direzione dello spazio \textbf{(vedi 3
Novembre?)} e descrive come varia la concentrazione del soluto rispetto
alla concentrazione iniziale grazie al processo di diffusione. I flussi
si misurano in moli di soluto su secondi per m$^2$. Questi sono fenomeni
che funzionano bene su distanze piccole e male su distanze macroscopiche
poichè distanza e tempo sono legate da una relazione quadratica.

Questo ci porta a capire come mai i movimenti dell'acqua nelle piante
non sono determinati dal fenomeno di diffusione. I potenziali chimici
sono funzioni termodinamiche che dicono quanta energia c'è in una data
sostanza. Questo dipende da vari fattori

\textbf{$\mu$$_j$= $\mu$$_j$$_0$ + RtlnC + PV + m$_w$gh + z$_j$ FE}

\textbf{(CONTROLLARE)}

Dove:

\begin{itemize}
\itemsep1pt\parskip0pt\parsep0pt
\item
  $\mu$$_j$$_0$ è la costante specifica della sostanza;
\item
  RtlnC è il logaritmo della concentrazione;
\item
  PV è la pressione per volume di j;
\item
  m$_w$gh è la disposizione in altezza;
\item
  z$_j$ FE è la carica della molecola.
\end{itemize}

Il movimento dell'acqua avviene da potenziali alti a potenziali più
bassi. (vedi quaderno)

L'acqua è una molecola caratterizzata da dipoli (frazioni di carica
negativa o positiva). Le molecole infatti interagiscono tra loro. Il
calore latente di evaporazione dell'acqua è molto alto e quindi la
traspirazione dell'acqua da parte delle piante è un processo favorito ed
efficiente per smaltire il calore. L'acqua, grazie ai suoi dipoli,
interagisce con altre molecole di acqua secondo i fenomeni di coesione.
Questa è la tendenza dell'acqua ad interagire con se stessa e ad
interagire con le superfici che la circondano (per esempio ad interagire
con la superficie delle parete dei vasi xilematici).

Grazie all'interazione di queste forze si forma una sorta di filo di cui
si può misurare la tensione. È una tensione molto forte. L'acqua entra
nelle piante perché hanno gradiente diffusibile favorevole. Il
potenziale idrico dell'acqua nel terreno influenza la presenza della
stessa nel terreno.

L'acqua è una molecola che passa sia attraverso le membrane biologiche
che attraverso canali specifici quali le \textbf{acquaporine}. Grazie
alle acquaporine passa molto più velocemente. Le acquaporine sono
elementi ad altissima conducibilità mentre le membrane hanno una
conducibilità bassa. La conducibilità complessiva dipende da entrambe.
Nella membrana la conducibilità dipende dai fosfolipidi mentre nelle
aquaporine il flusso può essere regolato dalla pianta mediante
fosforilazione. La conducibilità complessiva è data dalla somma dei due
tipi di conducibilità.

L'acqua entra preferenzialmente da alcune zone della radice in cui è
presente solamente la struttura primaria (soprattutto nei \emph{peli
radicali}).

L'acqua entra nei vasi addetti al suo trasporto tramite due vie:

\begin{enumerate}
\def\labelenumi{\arabic{enumi}.}
\itemsep1pt\parskip0pt\parsep0pt
\item
  una via \textbf{apoplastica};
\item
  una via \textbf{simplastica}.
\end{enumerate}

Nella via apoplastica le molecole d'acqua attraversano le cellule
passando dall'esterno della radice alla zona più interna diffondendo
attraverso le membrane o tramite le acquaporine. Nella zona più centrale
della radice tuttavia, si ha la presenza di uno strato di cellule
suberificate chiamate \textbf{``banda del Caspari''}. La superina che
riveste queste cellule le rende impermeabili e blocca la via apoplastica
facendo sì che l'acqua passi alla via simplastica. \textbf{(cercare via
simplastica)}

Le bande del Caspari hanno una relazione con la pressione radicale in
quanto impedisce la traspirazione. Passa per osmosi le bande del
caSPARI.

Quando si parla del trasporto di acqua all'interno del cilindro centrale
invece ci riferiamo a fenomeni di \emph{trasporto di massa}. Si parla di
trasporto di massa quando ci si riferisce al flusso di una soluzione
dentro un tubo in risposta ad una differenza di pressione idrostatica. I
flussi di massa sono regolati dalla legge del flusso di Poiseuille;
secondo questa legge la quantità di un liquido che passa in una sezione
dipende dal raggio della sezione e della viscosità del liquido.

J= $\pi$ r$^4$/8 $\eta$ x $\delta$P/$\delta$X

Dove:

\begin{itemize}
\itemsep1pt\parskip0pt\parsep0pt
\item
  \emph{r} è il raggio;
\item
  \emph{$\eta$} è la viscosità;
\item
  \emph{$\delta$P/$\delta$X} è il gradiente di pressione.
\end{itemize}

Il raggio è molto importante per la determinazione del flusso.
\textbf{(quaderno)}

L'evaporazione è un fenomeno passivo spontaneo quindi deve essere legata
per forza a qualcosa di presente nell'atmosfera. Nello specifico, ciò
che determina l'evaporazione è l'\emph{umidità}.

Cos'è il potenziale idrico dell'acqua nell'atmosfera? Si parla di
umidità relativa dell'aria per descrivere la quantità di acqua presente
nell'aria. Questa percentuale varia di temperatura in temperatura, e i
livelli possono essere \emph{sottosaturanti} o \emph{sovrasaturanti} in
base ad essa. Si ha una saturazione del 100\% appena prima che l'acqua
condensi. L'umidità relativa può assumere valori che vanno da 0 a 100.
L'umidità relativa (ovvero l'acqua presente sotto forma di vapore acqueo
nell'atmosfera) e quella assoluta interagiscono fortemente nel
determinare quanto una pianta traspira: se l'umidità relativa è bassa
l'acqua evapora più facilmente.

Per determinare il potenziale idrico dell'acqua nell'aria ha a che fare
con l'umidità relativa in base a quella assoluta. Se si ha umidità
relativa molto bassa il potenziale idrico diventa molto negativo. Se il
rapporto è zero la funzione tende a meno infinito. \textbf{(vedo formula
su quaderno)}

L'acqua sale nelle piante per un flusso di massa che dipende dalla
differenza di pressione idrostatica generata dall'acqua che passa dalla
fase liquida. Il metabolismo C4 è un adattamento alla mancanza di acqua.
Gli adattamenti delle piante agli ambienti vengono definiti nel
complesso \textbf{habitus xeromorfo}. Le piante sono dette xeromorfe. Le
piante acquatiche secondarie sono piante uscite dall'acqua e poi
ritornate in acqua. Per queste piante è difficile traspirare. La CO$_2$
entra nelle piante per diffusione casuale. In aria la sua diffusione è
migliaia di volte più grande di quella in acqua. Per l'ossigeno vale lo
stesso. Il potenziale redox del terreno dove vivono queste piante è
quasi proibitivo infatti vivono quasi sempre in anaerobiosi. L'acqua
dunque diventa un problema anche in presenza di grandi quantità della
stessa. I parenchimi aeriferi permettono di accumulare gas in quelle
piante soggette ad allagamento.

Le piante non hanno sistemi di regolazione per la temperatura e per
questo motivo tendono a seguire la temperatura dell'ambiente.

Nel mesofillo fogliare ci sono spazi aeriferi che si riempiono di vapore
acqueo. Se varia la temperatura il livello di saturazione cambia.
Traspirano maggiormente in presenza di alte temperature.

\subsubsection{Galla del colletto}\label{galla-del-colletto}

Le piante possono essere soggette ad agenti eziologici. Il primo virus
ad essere stato scoperto nella storia della biologia è stato quello del
mosaico del tabacco, ed è stato scoperto nelle piante. I principali
agenti eziologici delle piante sono i funghi.

Tuttavia le piante hanno una grande capacità di difesa in natura. Le
malattie colpiscono maggiormente le piante coltivate.

Le piante da frutto soffrono di una malattia chiamata \textbf{``galla
del colletto''}. Il colletto è la zona alla base della piante dove si
inverte la percezione della gravità (stacco tra radice e tronco).

Le malattie si manifestano spesso in seguito ad una ferita. L'infezione
da parte di questi virus non causa la morte della pianta ma è come se la
pianta producesse meno fotosintati.

Gli studi su questa malattia sono stati moltissimi, soprattutto a causa
del rilevante interesse economico. Più recentemente si è compreso che la
malattia è dovuta alla presenza di un batterio della famiglia delle
\textbf{Rizobiacee}. \textbf{Agrobacterium tumefaciens} è un batterio
contenente un cromosoma e un DNA extracromosomico plasmidiale che
conferisce caratteristiche di virulenza.

All'interno di una specie il DNA è trasmesso in maniera verticale.
Quando il trasferimento avviene tra una specie e un'altra invece, è
detto \emph{trasferimento orizzontale}.

Le piante se ferite secernono metaboliti come i fenoli e sostanze affini
che funzionano da \textbf{batteriostatici} (sostanze tossiche per i
batteri che ne impediscono la riproduzione). Agrobacterium, che è un
batterio ubiquitario nel terreno, è capace di risalire lungo gradienti
di concentrazione di queste sostanze. Nell'arco di qualche giorno sulla
zona del colletto, dove è presente la ferita, si formano delle galle. Le
\emph{galle} sono formate da tessuti che hanno proprietà particolari.
Questi tessuti hanno acquisito due facoltà importanti:

\begin{enumerate}
\def\labelenumi{\arabic{enumi}.}
\itemsep1pt\parskip0pt\parsep0pt
\item
  sono in grado di crescere in beute di agar e crescere in vitro in
  assenza di auxine e citochinine (che sono normali sostanze usate per
  stimolare la crescita dei vegetali in vitro);
\item
  producono una classe di molecole nuove che sono le \textbf{opine},
  ossia molecole composte da zuccheri e lipidi o zuccheri e amminoacidi,
  sono molecole estremamente ricche di energia.
\end{enumerate}

I frutti prodotti da piante infettate con Agrobacterium sono meno ricche
di zuccheri perché questi vengono in genere dirottati per la produzione
di opine.

La pianta infettata differenzia delle cellule che si mettono al servizio
del batterio; le galle diventano delle nicchie ecologiche finalizzate
alla nutrizione del batterio. La pianta stessa inizia a compire attività
nuove che prima dell'infezione del batterio non faceva. Le cellule della
galla sono cellule vegetali che sono state trasformate in seguito
all'infezione.

Il T-DNA del plasmide è definito da due \emph{``border'' (right e left)}
e contiene zone definite da geni che codificano per l'\emph{auxina},
geni che codificano per le \emph{citochinine}, e geni che codificano per
le \emph{opine}. Si parla di \textbf{T-DNA (DNA transfer)} perché viene
trasferito dal batterio alla cellula vegetale. Dopo il trasferimento,
nel plasmide rimangono i geni per il catabolismo delle opine, i geni
\emph{vir} per la virulenza e i geni per l'origine della replicazione.

\textbf{Octopine} e \textbf{nopaline} sono due differenti classi di
opine e definiscono due plasmidi distinti.

Dopo l'adesione del batterio alla superficie vegetale si ha un fenomeno
di \emph{coniugazione} tra la cellula eucariote e quella procariote.
Dall'interazione si origina un classico pilo (formato da proteine)
citoplasmatico attraverso il quale il T-DNA viene trasferito alle
cellula eucariotica. Una volta avvenuto il contatto, il plasmide viene
traslocato nel nucleo e si integra in una zona casuale del DNA nucleare
della cellula vegetale.

Il DNA acquisito può finire in zone caratterizzate da geni non
codificanti così come in zone caratterizzate da geni trascrivibili e
dunque si ha una trascrizione inserzionale. Il gene del T-DNA viene
trascritto perché il procariote è in grado di far esprimere il suo DNA
nel genoma eucariote. Questo è un meccanismo estremamente evoluto e
basato sulla sequenza di basi codoniche (si hanno codoni affini alla
trascrizione eucariotica).

I recettori di superficie della membrana cellulare adoperano una
trasduzione del segnale. Nei batteri i trasduttori sono sensori a due
componenti: sono dimeri proteici che, se interagiscono con il loro
ligando, si autofosforilano. In Agrobacterium, una volta che questo ha
interagito con un recettore ed il recettore è stato fosforilato, si ha
la fosforilazione dell'effettore da parte del recettore.

La proteina effettrice si stacca dal complesso e va a funzionare come
fattore di trascrizione di nuovi geni direttamente o attraverso altri
step. È la proteina \textbf{VirG} che una volta forforilata va ad agire
come fattore di trascrizione.

L'\textbf{acetilsiringone} invece è la molecola prodotta dalla pianta
ferita che si lega al recettore batterico e dà il via alla
fosforilazione. \emph{VirG} è responsabile del trasferimento del T-DNA
alla cellula vegetale ferita; questa proteina, una volta legata al sito
promotore, attua la trascrizione di un certo numero di veni \emph{vir}.
Tra i più importanti troviamo: \emph{virB}, una endonucleasi, e
\emph{virE}, una proteina che lega il DNA. Una volta prodotto virB,
\emph{virD} si lega e taglia il DNA plasmidico. VirD si lega ad una sola
estremità del DNA e dopo essere stato tagliato il DNA questo viene
srotolato da un'elicasi. Del DNA viene tagliato un solo filamento della
doppia elica, mentre l'altro rimane a livello plasmidiale e viene
risintetizzato. Il DNA a singolo filamento generalmente non è stabile e
per questo viene stabilizzato dalle proteine \emph{virE} che lo
proteggono dalla degradazione da parte della cellula ospite. Questo
singolo filamento va ad integrarsi nel genoma della pianta.

I geni espressi da questo filamento causano la formazione delle cellule
della galla del colletto che iniziano così a sintetizzare le opine.
Citochinine e auxine sono sostanze di crescita prodotte grazie al DNA
plasmidico importato nella cellula eucariote e permettono lo sviluppo
appunto di questo tipo cellulare.

Tramite la modifica del T-DNA si può sviluppare una tecnica
biotecnologica: E' possibile clonare un organismo tramite l'impianto del
nucleo di una sua cellula somatica all'interno di una cellula uovo
privata del suo DNA. Nelle piante questa tecninca viene effettuata da
molti anni tramite l'impianto di un protoplasto (cellula senza parete
cellulare); nell'arco di poco tempo si ha una pianta con un ciclo vitale
completo (in presenza di condizioni adatte al suo sviluppo). Se si
utilizza una cellula il cui DNA contiene una mutazione, allora tutte le
cellule del nuovo individuo che si svilupperà conterranno quella
mutazione (gameti compresi).

Gli enzimi di restrizione e quelli trascrizionali permettono di
costruire qualsiasi plasmide.

Esempi di piante trasformate geneticamnete:

\begin{itemize}
\itemsep1pt\parskip0pt\parsep0pt
\item
  \textbf{piante antisenso} per la \emph{poligalatturonidasi}. Se si
  esprime l'antisenso si forma un RNA messaggero complementare al senso
  e quindi in grado di formare una doppia elica insieme al filamento
  senso normalmente trascritto. L'espressione del filamento antisenso di
  un gene porta all'annullamento della trascrizione dell'RNA (è una
  forma di silenziamento genico). L'espressione delle
  poligalatturonidasi ha a che fare con quanto un frutto è vendibile per
  il suo grado di maturazione. Questa espressione viene silenziata per
  impedirne una maturazione precoce e per far sì che il frutto sia
  vendibile per un periodo di tempo più lungo;
\item
  \textbf{bioinsetticidi}. Inizialmente si usava il \emph{piombo
  arsenato} ma questi elementi creavano problemi anche alla salute
  umana. Si passò al \emph{DDT}, ma si scoprì che anch'esso era
  pericoloso. Attualmente si usano gli \emph{organo fosfati}. Questi
  sono fosfati organici che bloccano le giunzioni a livello delle
  neurosinapsi, bloccando così il rilascio dell'acetilcolina. Questo,
  negli insetti, impedisce il movimento delle ali. L'alternativa sono
  sostanze di origine vegetale provenienti dal metabolismo secondario:
  molecole tipo il piretro (zampirone), il rotenone e la nicotina (è
  l'insetticida più potente che esista in natura). L'altra via è quella
  dei \emph{biopesticidi}, ovvero l'utilizzo di organismi patogeni per
  gli insetti. Un esempio famoso è quello di \emph{Bacillus
  Thuringensis}, che è estremamente pericoloso per gli insetti. Questo è
  un batterio aerobio che in presenza di condizioni ambientali
  sfavorevoli entra in sporulazione e differenzia un \emph{cristallo
  parasporale}. Questi cristalli sono composti da proteine e sono
  estremamente tossici per gli insetti. Quando l'insetto morde la
  foglia, rompe le cellule ed ingoia questi cristalli, le spore fiiscono
  nel suo stomaco, dove si ha un pH basico. Le proteine che
  costituiscono il cristallo, in queste condizioni, diventano tossiche
  in quanto formano dei buchi nella membrana plasmatica delle cellule
  dell'epitelio. Basta una proteina per uccidere una cellula perciò il
  livello di tossicità è elevatissimo. Queste proteine sono innocue per
  l'uomo. Il 90\% del cotone è prodotto da piante che producono da sole
  questa proteina. Gli insetticidi bloccano il trasporto degli elettroni
  e dunque l'assimilazione di CO$_2$, oppure bloccano l'ammino sintetasi
  e ciò blocca l'assimilazione dell'ammonio. Hanno target specifici
  facili da ingegnerizzare.
\end{itemize}

Le galle del colletto sono strutture anche parecchio grandi. Siamo in
grado di usare Agrobacterium per trasferire nuovi geni e costruire
organismi che in natura non esisterebbero. Bisogna portare la mutazione
nella linea germinale per fare sì che la mutazione si trasmetta.

Circa 12000 anni fa, nel Medio Oriente, si è iniziata a sviluppare
l'agricoltura. Il bacino del mediterraneo è tipico per la produzione di
grano. Via via che le condizioni climatiche lo hanno permesso gli uomini
sono migrati in europa portandosi dietro questa coltivazione. È
importante poter produrre cibo in loco per poter cambiare vita, poichè
in questo modo si possono produrre quantità di cibo sufficienti per il
sostentamento della popolazione.

Per compiere un'agricoltura ``sensata'' è necessario conoscere e
selezionare le piante, poichè molte di esse contengono anche tantissime
sostanze nocive che usano come forma di difesa. Si devono dunque
selezionare le piante che non producono o producono soltanto una
piccolissima quantità di queste sostanze.

La Soia è un alimento difficile da digerire, poichè normalmente
producono una proteina che blocca la produzione della \emph{tripsina}.
Per questo motivo sono state selezionate alcune varietà che contengono
una bassa quantità di questa proteina e che sono dunque più digeribili.
Queste piante vengono isolate riproduttivamente per essere selezionate.

Il Teosinte è il progenitore del mais; questa è una pianta ramificata
che produce tanti fiori maschili. Il mais invece non è ramificato, ha un
solo fiore maschile e produce molti fiori femminili: le pannocchie. La
pannocchia del teosinte venne selezionata per portare alla pannocchia
attuale, molto più grande.

Le due piante differiscono per 6-7 geni; tra questi ne troviamo uno che
codifica per la ramificazione della pianta e per come è formata la
struttura su cui sono inseriti i chicchi (rachidi) della pannocchia.

Lo stesso discorso vale per i pomodori, per il frumento, ecc.

Brassica oleracea è la pianta che si trova alla base di cavolfiori,
broccoli, cavoletti di Bruxelles ecc. È sempre la stessa pianta ma è
stata selezionata per caratteri diversi.

``Addomesticare'' una pianta non è semplice, ed infatti è un processo
che storicamente non è avvenuto molte volte. L'addomesticazione di una
specie consiste nel dirottare i flussi genici e provvedere
all'isolamento riproduttivo. Le piante addomesticate hanno una miglior
fitness in termini di stima di quanto un organismo è adatto ad un
determinato ambiente in merito alla capacità di riprodursi e propagarsi.
Anche questi processi sono una forma di biotecnologia.

Se si vuole ottenere una pianta di grano ottimale, questa deve essere
bassa e con la spiga grande. Se incrocio una pianta alta e una bassa si
ottengono piante di vario tipo. Se ne seleziona una ottimale e si
continua a reincrociare fino a che non si ottengono individui con le
caratteristiche desiderate. Se si incrociano queste spighe tra di loro
si riesce a fissare il dato carattere nella varietà. Se i caratteri
dipendessero da un singolo gene nel giro di poco si riuscirebbe a
selezionare un carattere.

L'endosperma del riso è un alimento ricco di carboidrati ma carente di
provitamina A. Il riso per essere conservato deve subire il processo di
brillatura. La carenza di provitamina A può portare all'insorgenza di
alcune malattie come la xeroftalmia (porta alla cecità a causa
dell'assenza di lacrimazione, che a sua volta causa danni alla retina).
Sono soggetti a questa patologia soprattutto bambini e donne in
gravidanza. In cina aggiungono vitamina A al riso. È possibile esprimere
alcuni geni per la sintesi di questa vitamina dentro il riso; questa
modificazione fa sì che il riso assuma un colore giallo (per questo
chiamato \emph{golden rice}).

\end{document}
